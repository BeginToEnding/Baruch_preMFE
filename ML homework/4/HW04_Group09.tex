\documentclass{article}
\usepackage{amsmath} % for matrices
\usepackage{amssymb}
\usepackage{enumitem}
\usepackage{booktabs} % For better looking tables
\usepackage{graphicx} % For including images
\usepackage{caption}  % (Optional) For customizing captions
\usepackage{siunitx}
\usepackage{pdfpages}
\PassOptionsToPackage{hyphens}{url}\usepackage{hyperref}

\title{Baruch ML HW 4}
\author{Annie Yi, Daniel Tuzes, group 9}

\begin{document}
\maketitle
\section{exercise}
In a Fama-French 5-factor model, the expected return of a stock is given by the equation:
\begin{align*}
    \mathbf R_i - \mathbf R_f = \alpha_i \mathbf 1+ \beta_{i,R_M} (\mathbf R_M - \mathbf R_f) & + \beta_{i,\text{SMB}} \cdot \mathbf{SMB} +                    \\
                                                                                              & + \beta_{i,\text{HML}} \cdot \mathbf{HML} +                    \\
                                                                                              & + \beta_{i,\text{RMW}} \cdot \mathbf{RMW} +                    \\
                                                                                              & + \beta_{i,\text{CMA}} \cdot \mathbf{CMA} + \mathbf \epsilon_i
\end{align*}
where:

\begin{itemize}
    \item $\mathbf R_i$ is the return of stock $i$. We will use Apple's return using Yahoo's financial services.
    \item $\mathbf R_f$ is the risk-free rate, $\mathbf R_M$ is the market return
          (measured by all companies appearing in CRSP, weighted by their market capitalization)
    \item $\alpha_i$ is the intercept of the regression, representing the idiosyncratic return of stock $i$.
    \item $\beta_{i,R_M}$ is the sensitivity of stock $i$ to the market return.
    \item $\beta_{i,\text{SMB}}$, $\beta_{i,\text{HML}}$, $\beta_{i,\text{HML}}$ and $\beta_{i,\text{CMA}}$
          are the sensitivity of stock $i$ to the
          SMB (Small Minus Big),
          HML (High Minus Low),
          RMW (Robust Minus Weak) and
          CMA (Conservative Minus Aggressive) factors.
\end{itemize}
The bold symbols are vectors, having values for different time points.
We want to minimise it in the norm of $\mathbf \epsilon_i$ by
doing an OLS fitting to every stock $i$.

From a webpage \footnote{\url{http://mba.tuck.dartmouth.edu/pages/faculty/ken.french/data_library.html}}
we can download the data \footnote{\url{http://mba.tuck.dartmouth.edu/pages/faculty/ken.french/ftp/F-F_Research_Data_5_Factors_2x3_daily_CSV.zip}}
containing the factors Fama-French 5-factor model.
\begin{table}[ht]
    \caption{Fama-French 5-factor data tail}
    \label{FamaFrench}
    \begin{tabular}{lllllll}
        \toprule
                 & Mkt-RF & SMB   & HML   & RMW   & CMA   & RF    \\
        \midrule
        20241224 & 1.11   & -0.12 & -0.05 & -0.13 & -0.37 & 0.017 \\
        20241226 & 0.02   & 1.09  & -0.19 & -0.44 & 0.35  & 0.017 \\
        20241227 & -1.17  & -0.44 & 0.56  & 0.41  & 0.03  & 0.017 \\
        20241230 & -1.09  & 0.24  & 0.74  & 0.55  & 0.14  & 0.017 \\
        20241231 & -0.46  & 0.31  & 0.71  & 0.33  & 0.0   & 0.017 \\
        \bottomrule
    \end{tabular}
\end{table}

Using Yahoo finance, we can get the returns for a selected stock. As Apple was not available in the
Yahoo finance API, we used the stock of Microsoft. Some of the columns for the last few days are shown in
table \ref{MSFT}.

\begin{table}
    \caption{Microsoft stock data tail}
    \label{MSFT}
    \begin{tabular}{llllll}
        \toprule
                 & Open    & High    & Low     & Close   & return \\
        Date     &         &         &         &         &        \\
        \midrule
        20250225 & 401.100 & 401.920 & 396.700 & 397.900 & -0.015 \\
        20250226 & 398.010 & 403.600 & 394.250 & 399.730 & 0.005  \\
        20250227 & 401.270 & 405.740 & 392.170 & 392.530 & -0.018 \\
        20250228 & 392.660 & 397.630 & 386.570 & 396.990 & 0.011  \\
        20250303 & 398.820 & 398.820 & 386.160 & 388.490 & -0.021 \\
        \bottomrule
    \end{tabular}
\end{table}

From here, we can calculate the excess return of the stock by subtracting the risk-free rate from the stock's return.
Note that the risk-free rate is given in percentage, so we need to divide it by 100.
As $X$, we will use the market's excess return, the SMB, HML, RMW and CMA factors.
As $y$, we will use the excess return of the stock.
By doing an OLS fitting, we can get $\alpha_i$ and the betas,
which are the coefficients of the regression SMB, HML, RMW and CMA.

The output of the regression is
\footnotesize
\begin{verbatim}
Intercept: 0.00020720675127399466
Coefficients: [ 0.01128495 -0.00369999 -0.00378986  0.00249309 -0.00161599]
R-squared: 0.6079755982101198
\end{verbatim}
\normalsize

Using the coefficients, we can calculate the predicted return of the stock
as a function of the actual return as shown in
figure \ref{fig:predicted_return}.
% include the figure
\begin{figure}[ht]
    \centering
    \includegraphics[width=0.8\textwidth]{predicted_vs_actual_excess_return.pdf}
    \caption{Predicted return of the stock as a function of the actual return}
    \label{fig:predicted_return}
\end{figure}

We can also show the cumulative excess return of the stock and
the predicted cumulative excess return of the stock as a function of time as shown in
figure \ref{fig:cumulative_return}. Note that the cumulative excess returns
may look very similar, even for models with low R-squared values.

\begin{figure}[ht]
    \centering
    \includegraphics[width=0.8\textwidth]{cumulative_excess_return.pdf}
    \caption{Cumulative excess return of the stock and
        the predicted excesscumulative return of the stock}
    \label{fig:cumulative_return}
\end{figure}

Below the jupyter notebook used to generate this content, as pdf, can be found.
To generate the pdf from the jupyter notebook, we used the nbconvert tool.
The commands used were:
\begin{verbatim}
jupyter nbconvert --to latex fitting.ipynb
pdflatex fitting.tex
\end{verbatim}
\includepdf[pages=-]{fitting.pdf}

\end{document}
