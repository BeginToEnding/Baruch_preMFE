\documentclass{article}
\usepackage{amsmath} % for matrices
\usepackage{amssymb}
\usepackage{enumitem}
\usepackage{booktabs} % For better looking tables
\usepackage{graphicx} % For including images
\usepackage{caption}  % (Optional) For customizing captions
\usepackage{siunitx}
\usepackage{pdfpages}
\PassOptionsToPackage{hyphens}{url}\usepackage{hyperref}

\title{Baruch ML HW 5}
\author{Annie Yi, Daniel Tuzes, group 9}

\begin{document}
\maketitle
\section*{Predicting the Consumer Price Index}
To predict the Consumer Price Index, we build a linear model containing features as below.
We are excited to see which one of these will be have lower weights in the regression,
and didn't want to neglect them before seeing the model in action.
However, we have opionions based on which is caused by the inflation,
and which one of these may have low correlation with the results,
see the section discussion.
\begin{enumerate}
    \item Past CPI Values: Historical CPI data to capture the persistence of inflation.
          Source: U.S. Bureau of Labor Statistics (BLS),
          Consumer Price Index for All Urban Consumers (CPI-U), U.S. city average, All items - CUUR0000SA0,
          \url{https://data.bls.gov/toppicks?survey=cu}
    \item Output Gap: The difference between actual and potential economic output. A positive output gap can
          lead to higher inflation. Source: International Monetary Fund (IMF)
    \item Unemployment Rate: Lower unemployment rates can lead to higher inflation based on the Phillips
          Curve. Source: U.S. Bureau of Labor Statistics (BLS)
    \item Interest Rates: Central bank policy rates, such as the Federal Funds Rate, influence inflation through
          monetary policy. Source: Federal Reserve Economic Data (FRED). \url{https://fred.stlouisfed.org/series/FEDFUNDS}
    \item Money Supply: Measures like M2 (a broad measure of money supply) can impact inflation. Source:
          Federal Reserve Board.
    \item Wage Growth: Increases in wages can lead to higher consumer spending and inflation.
          Source: Federal Reserve Bank of Atlanta.
          \url{https://www.atlantafed.org/chcs/wage-growth-tracker}, column "Overall"
    \item Commodity Prices: Prices of key commodities like oil and food can directly affect inflation.
          Source:
          International Monetary Fund (IMF). \url{https://data.imf.org/?sk=471dddf8-d8a7-499a-81ba-5b332c01f8b9&sid=1547558078595}
          All commodities index
    \item Exchange Rates: Changes in exchange rates can influence import prices and thus inflation. Source:
          Federal Reserve Board
\end{enumerate}
we collected data from
\end{document}
