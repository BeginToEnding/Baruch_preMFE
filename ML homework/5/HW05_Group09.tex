\documentclass{article}
\usepackage{amsmath} % for matrices
\usepackage{amssymb}
\usepackage{enumitem}
\usepackage{booktabs} % For better looking tables
\usepackage{graphicx} % For including images
\usepackage{caption}  % (Optional) For customizing captions
\usepackage{siunitx}
\usepackage{pdfpages}
\PassOptionsToPackage{hyphens}{url}\usepackage{hyperref}

\title{Baruch ML HW 5}
\author{Annie Yi, Daniel Tuzes, group 9}

\begin{document}
\maketitle
\section*{Predicting the Consumer Price Index}
To predict the Consumer Price Index, we build a linear model containing features as below.
We are excited to see which one of these will be have lower weights in the regression,
and didn't want to neglect them before seeing the model in action.
However, we have opionions based on which is caused by the inflation,
and which one of these may have low correlation with the results,
see the section discussion.
\begin{enumerate}
      \item Past CPI Values: Historical CPI data to capture the persistence of inflation.
            Source: U.S. Bureau of Labor Statistics (BLS),
            Consumer Price Index for All Urban Consumers (CPI-U), U.S. city average, All items - CUUR0000SA0,
            \url{https://data.bls.gov/toppicks?survey=cu}
      \item Output Gap (OG): The difference between actual and potential economic output.
            A positive output gap can lead to higher inflation. Source: IMF,
            \url{https://www.imf.org/en/Publications/WEO/weo-database/2024/October/download-entire-database}
            Note that data after 2023 are not measured but predicted,
            and data frequency is in years, not months.
      \item Unemployment Rate (UR): Lower unemployment rates can lead to higher inflation based on the Phillips
            Curve. Source: U.S. Bureau of Labor Statistics (BLS), \url{https://data.bls.gov/toppicks?survey=ln},
            Unemployment Rate - LNS14000000
      \item Interest Rates (IR): Central bank policy rates, such as the Federal Funds Rate, influence inflation through
            monetary policy. Source: Federal Reserve Economic Data (FRED). \url{https://fred.stlouisfed.org/series/FEDFUNDS}
      \item Money Supply (M2SL): Measures like M2 (a broad measure of money supply) can impact inflation. Source:
            Federal Reserve Board. \url{https://fred.stlouisfed.org/series/M2SL}
      \item Wage Growth (WG): Increases in wages can lead to higher consumer spending and inflation.
            Source: Federal Reserve Bank of Atlanta.
            \url{https://www.atlantafed.org/chcs/wage-growth-tracker}, column "Overall"
      \item Commodity Prices (CI): Prices of key commodities like oil and food can directly affect inflation.
            Source:
            International Monetary Fund (IMF). \url{https://data.imf.org/?sk=471dddf8-d8a7-499a-81ba-5b332c01f8b9&sid=1547558078595}
            All commodities index
      \item Exchange Rates: Changes in exchange rates can influence import prices and thus inflation. Source:
            Federal Reserve Board,
            Nominal Broad Dollar Index, H10/H10/JRXWTFB\_N.B
            \url{https://www.federalreserve.gov/datadownload/Download.aspx?rel=H10&series=122e3bcb627e8e53f1bf72a1a09cfb81&lastObs=&from=01/01/2010&to=02/01/2025&filetype=csv&label=include&layout=seriescolumn}
      \item Consumer Confidence Index (CCI): Measures consumer sentiment and expectations about the economy.
            Higher confidence can lead to increased spending and inflation. Source: OECD.
            This feature was added later, and only its outcome is discussed.
            Calculation can be found in the attached notebook.
            \url{https://www.oecd.org/en/data/indicators/consumer-confidence-index-cci.html?oecdcontrol-cf46a27224-var1=USA}
\end{enumerate}


\begin{table}
      \caption{CPI and features are sourced starting from January 2015 till January 2025. The tables shows the first few entries.}
      \begin{tabular}{lrrrrrrrr}
            \toprule
            date   & CPI     & IR   & WG  & CI      & OG     & UR  & M2     & ER      \\
            \midrule
            Jan-20 & 216.687 & 0.11 & 1.6 & 139.84  & -2.326 & 9.8 & 8478   & 92.3566 \\
            Feb-20 & 216.741 & 0.13 & 1.7 & 136.553 & -2.326 & 9.8 & 8527.6 & 93.7321 \\
            Mar-20 & 217.631 & 0.16 & 1.9 & 141.525 & -2.326 & 9.9 & 8523.7 & 93.7979 \\
            Apr-20 & 218.009 & 0.2  & 1.9 & 149.313 & -2.326 & 9.9 & 8555.1 & 92.6661 \\
            May-20 & 218.178 & 0.2  & 1.6 & 140.402 & -2.326 & 9.6 & 8609.3 & 92.8423 \\
            \bottomrule
      \end{tabular}
\end{table}


\begin{figure}[ht]
      \centering
      \includegraphics[width=\linewidth]{data_raw.pdf}
      \caption{Economic Indicators Over Time.
            Note the consistent decrease in many of the indices since or around 2022,
            but by nature, we know that CPI can hardly decrease.}
      \label{fig:economic_indicators}
\end{figure}

The collected data first checked for outliers and missing data points by visual inspection.
Web observed no DQ issues apart from missing data.
Missing data is filled by either backward or forward filling
and differences in frequencies are resolved by resampling (backfilling).
We can see that most of the indicators went up around 2022,
and decreased since then a bit, while CPI kept increasing.
This changing trend cannot be captured by the model,
since this is observable only for recent year.
We expect the model to not to be able to predict well after 2022.

We did the linear regression, ridge and lasso regression at different alphas
on the first 12 years of data from 2010 till 2022, and inspected the output on the whole time range,
which includes data from end of 2022 till 2025. Plotted results are shown in the figures \ref{fig:ridge_predictions} and \ref{fig:lasso_predictions}.

\begin{figure}[ht]
      \centering
      \includegraphics[width=\linewidth]{data_pred_ridge.pdf}
      \caption{Predictions of the models with ridge regression.
            Training data ends at end of 2022.}
      \label{fig:ridge_predictions}
\end{figure}


\begin{figure}[ht]
      \centering
      \includegraphics[width=\linewidth]{data_pred_lasso.pdf}
      \caption{Predictions of the models with lasso regression.
            Training data ends at end of 2022.}
      \label{fig:lasso_predictions}
\end{figure}

The fitted paramters and R-squared are shown in the table \ref{tab:ridge_coefficients}.

\begin{table}
      \caption{Coefficient values and their names for ridge regression. in.cpt: axis intercept, lin reg: linear regression, rid: ridge regression, las: lasso regression}
      \label{tab:ridge_coefficients}
      \begin{tabular}{lrrrrrrr}
            \toprule
                   & Lin Reg & rid $\alpha_{0.1}$ & rid $\alpha_{1}$ & rid $\alpha_{10}$ & las $\alpha_{0.2}$ & las $\alpha_{1}$ & las $\alpha_{5}$ \\
            \midrule
            IR     & 15.27   & 13.7               & 8.587            & 4.296             & 7.1                & 0                & 0                \\
            WG     & 17.41   & 15.53              & 15.05            & 13.23             & 19.06              & 1.029            & 0                \\
            CI     & 14.42   & 15.22              & 12.43            & 6.074             & 4.976              & 0                & 0                \\
            OG     & -2.206  & -1.533             & 2.732            & 8.932             & 2.403              & 3.091            & 0                \\
            UR     & -3.361  & -4.33              & -5.727           & -5.138            & -0                 & -0               & -0               \\
            M2     & 48.4    & 45.7               & 36.11            & 21.41             & 45.83              & 49.71            & 7.824            \\
            ER     & 1.544   & 5.574              & 12.14            & 12.02             & 0                  & 0                & 0                \\
            CCI    & 4.729   & 3.165              & -3.095           & -6.137            & -0                 & -0               & -0               \\
            in.cpt & 212.5   & 213.1              & 217.8            & 225.7             & 218.1              & 224.5            & 242.6            \\
            $R^2$  & 0.9927  & 0.9925             & 0.9864           & 0.9216            & 0.9842             & 0.9176           & 0.2161           \\
            MSE    & 2.83    & 2.921              & 5.275            & 30.38             & 6.111              & 31.9             & 303.6            \\
            \bottomrule
      \end{tabular}
\end{table}
From the ridge and lasso regression,
we can see the differences in how differently they reduced the feature coefficients:
lasso regression eliminates the features, while ridge regression reduces them.

The parameter $\alpha$ has more severe effect on MSE and $R^2$ in lasso regression,
which is also reflected in the worsening fitting. To have a similar severity,
one needs to lower the $\alpha$ for the lasso regression.

From an econometrics point of view, we see the following:
\begin{enumerate}
      \item  money supply (M2) and wage growth (WG) are the most important features
      \item while ouput gap (OG) is not eliminated by lasso regression,
            it changes its sign depending on the strength and type of regularization
      \item interest rate (IR), suprisingly, lost its weight in the ridge regression,
            and we lost it completely at moderate and strong lasso regression
      \item unemployment (UR) and exchange rates (ER) had low weights,
            and they got eliminated by the weaker lasso regression
      \item Commodity prices (CI) played a consistent, but not too strong role
\end{enumerate}

It is surprising, how well the data can be fit till 2022, but breaks down after that.
By extending the fitting regime till 2025, the fitted curve still wouldn't align well with the
period 2022-2025.

\section*{Discussion}
The CPI can be well-captured by the model in a limited period of time,
but the model fails to predict properly after 2022. Till 2022,
only 3 features would be enough to capture most part of the curve,
$R^2 = 0.97$, see attached notebook and coefficients in table \ref{tab:ridge_coefficients_small}.


\begin{table}
      \caption{Coefficient values and their names for ridge regression.
            The model is shrunk to 3 features.
            in.cpt: axis intercept, lin reg: linear regression, rid: ridge regression, las: lasso regression}
      \label{tab:ridge_coefficients_small}
      \begin{tabular}{lrrrrrrr}
            \toprule
                   & Lin Reg & rid $\alpha_{0.1}$ & rid $\alpha_{1}$ & rid $\alpha_{10}$ & las $\alpha_{0.2}$ & las $\alpha_{1}$ & las $\alpha_{5}$ \\
            \midrule
            WG     & 30.38   & 29.99              & 27.37            & 17.75             & 24.52              & 1.029            & 0                \\
            OG     & 3.245   & 3.53               & 5.64             & 11.26             & 3.215              & 3.091            & 0                \\
            M2     & 42.02   & 41.77              & 39.42            & 25.87             & 43.55              & 49.71            & 7.824            \\
            in.cpt & 218.4   & 218.5              & 219.2            & 224.6             & 219.6              & 224.5            & 242.6            \\
            $R^2$  & 0.974   & 0.9739             & 0.9711           & 0.8826            & 0.9717             & 0.9176           & 0.2161           \\
            MSE    & 10.08   & 10.1               & 11.18            & 45.46             & 10.95              & 31.9             & 303.6            \\
            \bottomrule
      \end{tabular}
\end{table}


It is instructive to see that increasing the number of paramters
does not necessarily improve the quality of prediction.
The model is limited by its linear nature two fold:
\begin{enumerate}
      \item it cannot capture the time delay of the features,
            e.g. interest rate and money supply have a time delay effect on CPI (with different time scales),
      \item it cannot capture non-linear effects, e.g. the unemployment rate,
            which is alredy a weak feature, and even its supposedly effect is not linear
\end{enumerate}

We expected to a have a larger weight on interest rate,
as it is a direct tool of the Federal Reserve to control inflation,
but this regression didn't conculde that. We believe that bringing in more features
can not help in giving more meaningful predictions without building a more thourough financial model,
because either the changes in the new features are coused by or directly related to the inflation
(like a subset of goods, different basket, etc),
and they are not causing the inflation (however, the goal of such model would be to predict it),
or the time delay of the features cannot be captured by such a linear model.

However, CCI, the costumer confidence index, may be a good candidate in predicting the CPI.
Such parameter relies rather on a psychological factor,
and the expectations of the consumers and corporates may become self-fulfilling.
However, we cannot know how strong effect of the existing CPI prediction models
and their advertised outcome has on the CCI.

After obtaining the data and completing the regressions,
we have seen that the CPI is not a good predictor of inflation,
see the table \ref{tab:ridge_coefficients_CCI}.
\begin{table}
      \caption{Coefficient values and their names for ridge regression}
      \label{tab:ridge_coefficients_CCI}
      \begin{tabular}{lrrrrrrr}
            \toprule
                   & Lin Reg & rid $\alpha_{0.1}$ & rid $\alpha_{1}$ & rid $\alpha_{10}$ & las $\alpha_{0.2}$ & las $\alpha_{1}$ & las $\alpha_{5}$ \\
            \midrule
            CCI    & 4.729   & 3.165              & -3.095           & -6.137            & -0                 & -0               & -0               \\
            in.cpt & 212.5   & 213.1              & 217.8            & 225.7             & 218.1              & 224.5            & 242.6            \\
            $R^2$  & 0.9927  & 0.9925             & 0.9864           & 0.9216            & 0.9842             & 0.9176           & 0.2161           \\
            \bottomrule
      \end{tabular}
\end{table}

A good model relies more on a carefully selected set of features,
and an appropiate modelling approach, than on the number of features.


% incldue pdf
\includepdf[pages={1-}]{fitting.pdf}
\end{document}
