\documentclass{article}
\usepackage{amsmath}
\usepackage{amssymb}
\usepackage{array}
\usepackage{graphicx} % Required for \scalebox
\usepackage{hyperref}
\usepackage{listings}
\usepackage{xcolor}
\usepackage{graphicx}
\usepackage{booktabs}
\usepackage{verbatim}
\usepackage{adjustbox}
\usepackage{fancyvrb}
\setcounter{MaxMatrixCols}{20}
\title{Baruch NLA HW 7}
\author{Daniel Tuzes, 21}

\lstset{
    language=Python,
    basicstyle=\ttfamily\small,
    keywordstyle=\color{blue},
    commentstyle=\color{green!60!black},
    stringstyle=\color{orange},
    showstringspaces=false,
    breaklines=true,
    frame=single
}

\newenvironment{myverb}{%
 \VerbatimEnvironment
 \begin{adjustbox}{max width=\linewidth}
 \begin{BVerbatim}
  }{
  \end{BVerbatim}
 \end{adjustbox}
}

\DeclareMathOperator{\sgn}{sgn}

\begin{document}
\maketitle
\section{exercise}
\subsection*{i}
Given the regression model to estimate the 3-year Treasury bond yield \( T_3 \):
\[ T_3 = a + b_1 T_2 + b_2 T_5 + b_3 T_{10} \]

Where:
\begin{itemize}
    \item \( T_2, T_5, \) and \( T_{10} \) are the yields of the 2-year, 5-year, and 10-year Treasury bonds, respectively.
    \item \( a, b_1, b_2, b_3 \) are the coefficients to be estimated.
\end{itemize}

The matrix \( A \) and vector \( y \) are defined as:
\[ A = \begin{bmatrix}
        1      & T_{2,1} & T_{5,1} & T_{10,1} \\
        1      & T_{2,2} & T_{5,2} & T_{10,2} \\
        \vdots & \vdots  & \vdots  & \vdots   \\
        1      & T_{2,n} & T_{5,n} & T_{10,n}
    \end{bmatrix} \]
\[ y = \begin{bmatrix} a \\ b_1 \\ b_2 \\ b_3 \end{bmatrix} \]

The coefficients \( y \) are estimated using the
ordinary least squares method,
calculated as follows:
\[ y = (A^T A)^{-1} A^T T_3 \]
This formula ensures the minimization of the sum of
squared residuals between the observed values \( T_3 \)
and the predicted values \( A \cdot y \).
After fitting the model,
the estimated coefficients with $\text{error}_{\text{LR}} = 0.043$ are:
\[ y = \begin{bmatrix}
        -1.2720 \\
        0.1272  \\
        0.3340  \\
        0.5298
    \end{bmatrix} \]

The approximation error is calculated as the Euclidean norm of the residual vector, which is the difference between the observed values \( T_3 \) and the model's predictions \( T_{3,\text{LR}} = A \cdot y \). Mathematically, this is expressed as:
\[ \text{error}_{\text{LR}} = \|\ T_3 - T_{3,\text{LR}} \|\]
where \( T_{3,\text{LR}} \) is the vector of predicted \( T_3 \) values.

\subsection*{ii}

The linearly interpolated values are calcualted as
the linear combination of $T_2$ and $T_5$ with coefficients $2/3$ and $1/3$.

\begin{tabular}{rrrrrr}
    \toprule
    2Y   & 3Y   & 5Y   & 10Y  & $\text{3Y}_{\text{OLS Pred}}$ & $\text{3Y}_{\text{lin. interp.}}$ \\
    \midrule
    1.69 & 2.58 & 3.57 & 4.63 & 2.59                          & 2.32                              \\
    1.81 & 2.71 & 3.69 & 4.73 & 2.70                          & 2.44                              \\
    1.81 & 2.72 & 3.70 & 4.74 & 2.71                          & 2.44                              \\
    1.79 & 2.78 & 3.77 & 4.81 & 2.76                          & 2.45                              \\
    1.79 & 2.77 & 3.77 & 4.80 & 2.76                          & 2.45                              \\
    1.83 & 2.75 & 3.73 & 4.79 & 2.74                          & 2.46                              \\
    1.81 & 2.71 & 3.72 & 4.76 & 2.72                          & 2.45                              \\
    1.81 & 2.72 & 3.74 & 4.77 & 2.73                          & 2.45                              \\
    1.83 & 2.76 & 3.77 & 4.80 & 2.76                          & 2.48                              \\
    1.81 & 2.73 & 3.75 & 4.77 & 2.74                          & 2.46                              \\
    1.82 & 2.75 & 3.77 & 4.80 & 2.76                          & 2.47                              \\
    1.82 & 2.75 & 3.76 & 4.80 & 2.76                          & 2.47                              \\
    1.80 & 2.73 & 3.75 & 4.78 & 2.74                          & 2.45                              \\
    1.78 & 2.71 & 3.72 & 4.73 & 2.70                          & 2.43                              \\
    1.79 & 2.71 & 3.71 & 4.73 & 2.70                          & 2.43                              \\
    \bottomrule
\end{tabular}

This approximation gives $\text{error}_{\text{linear interp.}} = 1.099$.
\subsection*{iii}
We can do cubic spline for each day. On the first day, it would yield $2.42$, illustrated on \autoref{fig:spline_plot}.
\begin{figure}[h]
    \centering
    \includegraphics[width=0.8\textwidth]{sample_cspline.pdf}
    \caption{Piecewise cubic spline function based on 1st day's data}
    \label{fig:spline_plot}
\end{figure}

\begin{tabular}{rrrrrrr}
    \toprule
    2Y   & 3Y   & 5Y   & 10Y  & 3Y\_OLS & 3y\_lin\_interp & 3Y\_c\_spline \\
    \midrule
    1.69 & 2.58 & 3.57 & 4.63 & 2.59    & 2.32            & 2.39          \\
    1.81 & 2.71 & 3.69 & 4.73 & 2.70    & 2.44            & 2.51          \\
    1.81 & 2.72 & 3.70 & 4.74 & 2.71    & 2.44            & 2.51          \\
    1.79 & 2.78 & 3.77 & 4.81 & 2.76    & 2.45            & 2.53          \\
    1.79 & 2.77 & 3.77 & 4.80 & 2.76    & 2.45            & 2.53          \\
    1.83 & 2.75 & 3.73 & 4.79 & 2.74    & 2.46            & 2.53          \\
    1.81 & 2.71 & 3.72 & 4.76 & 2.72    & 2.45            & 2.52          \\
    1.81 & 2.72 & 3.74 & 4.77 & 2.73    & 2.45            & 2.53          \\
    1.83 & 2.76 & 3.77 & 4.80 & 2.76    & 2.48            & 2.55          \\
    1.81 & 2.73 & 3.75 & 4.77 & 2.74    & 2.46            & 2.53          \\
    1.82 & 2.75 & 3.77 & 4.80 & 2.76    & 2.47            & 2.54          \\
    1.82 & 2.75 & 3.76 & 4.80 & 2.76    & 2.47            & 2.54          \\
    1.80 & 2.73 & 3.75 & 4.78 & 2.74    & 2.45            & 2.52          \\
    1.78 & 2.71 & 3.72 & 4.73 & 2.70    & 2.43            & 2.50          \\
    1.79 & 2.71 & 3.71 & 4.73 & 2.70    & 2.43            & 2.50          \\
    \bottomrule
\end{tabular}

Now calculating the sum of the square of the difference between
the 3Y values and the cubic spline interpolated values, one gets
$\text{error}_{cubic spline} = 0.816$.

\section{exercise}
Calculate the mid prices and the difference $C-P$ as the first step.
Then from the book, eq. 8.22, we use

\[C - P = S{e^{ - qT}} - K{e^{ - rT}}.\]

\begin{tabular}{rrrrrrr}
    \toprule
    Bid\_Call & Ask\_Call & Bid\_Put & Ask\_Put & Strk & Mid\_Call & Mid\_Put \\
    \midrule
    259.30    & 260.70    & 34.80    & 35.70    & 2150 & 260.00    & 35.25    \\
    238.10    & 239.60    & 38.50    & 39.40    & 2175 & 238.85    & 38.95    \\
    217.40    & 218.90    & 42.50    & 43.50    & 2200 & 218.15    & 43.00    \\
    197.20    & 198.70    & 47.10    & 48.10    & 2225 & 197.95    & 47.60    \\
    177.40    & 178.90    & 52.10    & 53.20    & 2250 & 178.15    & 52.65    \\
    158.30    & 159.70    & 57.80    & 58.90    & 2275 & 159.00    & 58.35    \\
    139.80    & 141.30    & 64.10    & 65.30    & 2300 & 140.55    & 64.70    \\
    122.10    & 123.60    & 71.20    & 72.50    & 2325 & 122.85    & 71.85    \\
    105.30    & 106.80    & 79.10    & 80.50    & 2350 & 106.05    & 79.80    \\
    89.50     & 90.90     & 88.10    & 89.50    & 2375 & 90.20     & 88.80    \\
    74.80     & 76.10     & 98.10    & 99.50    & 2400 & 75.45     & 98.80    \\
    61.30     & 62.60     & 109.40   & 110.90   & 2425 & 61.95     & 110.15   \\
    49.10     & 50.30     & 122.00   & 123.50   & 2450 & 49.70     & 122.75   \\
    38.50     & 39.60     & 136.10   & 137.60   & 2475 & 39.05     & 136.85   \\
    29.40     & 30.30     & 151.70   & 153.30   & 2500 & 29.85     & 152.50   \\
    15.70     & 16.40     & 185.90   & 190.00   & 2550 & 16.05     & 187.95   \\
    7.70      & 8.20      & 227.20   & 231.70   & 2600 & 7.95      & 229.45   \\
    3.60      & 4.00      & 265.90   & 285.90   & 2650 & 3.80      & 275.90   \\
    1.75      & 2.05      & 313.70   & 333.60   & 2700 & 1.90      & 323.65   \\
    0.55      & 0.75      & 411.70   & 431.50   & 2800 & 0.65      & 421.60   \\
    \bottomrule
\end{tabular}


\subsection*{i}
Consider solving the linear system using the least squares technique:
\[
    A \mathbf{x} \approx \mathbf{y}
\]

Where:
\[
    A = \begin{bmatrix}
        1      & -K_1   \\
        1      & -K_2   \\
        1      & -K_3   \\
        \vdots & \vdots
    \end{bmatrix} =
    \begin{bmatrix}
        1      & -2150  \\
        1      & -2175  \\
        1      & -2200  \\
        \vdots & \vdots
    \end{bmatrix}, \quad
    \mathbf{y} = \begin{bmatrix}
        y_1 \\
        y_2 \\
        y_3 \\
        \vdots
    \end{bmatrix} =
    \begin{bmatrix}
        224.75 \\
        200.45 \\
        175.15 \\
        \vdots
    \end{bmatrix}
\]


The vector \( \mathbf{x} \) to be estimated contains:
\[
    \mathbf{x} = \begin{bmatrix}
        S \cdot e^{-qT} \\
        e^{-rT}
    \end{bmatrix}
\]

The least squares solution can be computed as:
\[
    \mathbf{x} = (A^T A)^{-1} A^T \mathbf{y}
\]

From \( \mathbf{x} \), knowing \( S \) and \( T \), one can compute \( q \) and \( r \) — the dividend yield and discount rate, respectively.


Given the outcomes from the least squares technique:
\[
    S \cdot e^{-qT} = 2360.17, \quad e^{-rT} = 0.9932
\]
with the spot price of the index \( S \) as 2381, and time to expiration \( T \), calculated from March 16, 2017, to September 29, 2017, is approximately:
\[
    T = 0.548 \text{ years}
\]
Substitute the values into the equations to solve for \( q \) and \( r \):
\[
    q = -\frac{\log\left(\frac{S \cdot e^{-qT}}{S}\right)}{T}, \quad r = -\frac{\log(e^{-rT})}{T}
\]

Using these formulas, we calculate:
\[
    q = -\frac{\log\left(\frac{2360.17}{2381}\right)}{T} = 1.60\%, \quad r = -\frac{\log(0.9931)}{T}=1.25\%
\]

\subsection*{ii}
By guessing the volatility, we get an implied price from BS.
Then the difference between the BS and the market price can be used to adjust the guessed volatility.
By implementing the book's methond of Table 8.3,
for the call with strike $2150$, we get the iterative steps:


\begin{tabular}{lrrrrrr}
    \toprule
      & $\sigma$ & d1      & d2      & C\_BS     & vega\_C   & $\Delta_\sigma$ \\
    \midrule
    0 & 0.250000 & 0.63369 & 0.44868 & 301.68231 & 570.02647 & 0.073123        \\
    1 & 0.176877 & 0.83037 & 0.69947 & 262.37153 & 493.59109 & 0.004805        \\
    2 & 0.172072 & 0.84995 & 0.72261 & 260.01913 & 485.53745 & 0.000039        \\
    \bottomrule
\end{tabular}


resulting in a final implied volatility of $0.172032$,
where the tolarance was 1e-6,
i.e. the last digit is inaccurate due to the interpolation.
Repeating the iteration for all the options, calls and puts, the results are:

\begin{tabular}{rrrrrr}
    \toprule
    Strike & Mid\_Call & Mid\_Put & impl call vol & impl put vol & diff [\%] \\
    \midrule
    2150   & 260.00    & 35.25    & 0.172         & 0.172        & 0.132     \\
    2175   & 238.85    & 38.95    & 0.167         & 0.167        & 0.152     \\
    2200   & 218.15    & 43.00    & 0.162         & 0.163        & 0.058     \\
    2225   & 197.95    & 47.60    & 0.158         & 0.158        & 0.025     \\
    2250   & 178.15    & 52.65    & 0.153         & 0.153        & 0.047     \\
    2275   & 159.00    & 58.35    & 0.148         & 0.148        & 0.069     \\
    2300   & 140.55    & 64.70    & 0.143         & 0.143        & 0.037     \\
    2325   & 122.85    & 71.85    & 0.138         & 0.138        & 0.06      \\
    2350   & 106.05    & 79.80    & 0.133         & 0.133        & 0.026     \\
    2375   & 90.20     & 88.80    & 0.129         & 0.129        & 0.003     \\
    2400   & 75.45     & 98.80    & 0.124         & 0.124        & 0.096     \\
    2425   & 61.95     & 110.15   & 0.119         & 0.119        & 0.075     \\
    2450   & 49.70     & 122.75   & 0.115         & 0.115        & 0.054     \\
    2475   & 39.05     & 136.85   & 0.11          & 0.11         & 0.173     \\
    2500   & 29.85     & 152.50   & 0.106         & 0.106        & 0.161     \\
    2550   & 16.05     & 187.95   & 0.099         & 0.098        & 1.147     \\
    2600   & 7.95      & 229.45   & 0.094         & 0.092        & 1.974     \\
    2650   & 3.80      & 275.90   & 0.091         & 0.093        & 1.96      \\
    2700   & 1.90      & 323.65   & 0.091         & 0.094        & 3.001     \\
    2800   & 0.65      & 421.60   & 0.096         & 0.10         & 4.224     \\
    \bottomrule
\end{tabular}

The implied volatilites go hand in hand,
and the relative difference is the lowest around the spot price.

\subsection*{iv}
Implement an explicit method to calculate implied volatility based on the \\article
\url{https://papers.ssrn.com/sol3/papers.cfm?abstract_id=2908494}.
The forward price of the asset at time $T$ can be calculated
from the fitted present value of the forward,
\[F=Se^{-qT} \cdot 1/e^{-rT} = 2376.41.\]
The rest can be calculated separately for calls and puts.
The vars $R$, $B$, $C$, $\gamma$, $\beta$, $\sigma$ are distinguished for puts and calls,
and $CP\_00 = Ke^{-rT}$.
Thanks to Kian, note that $A$ is both a variable and a function,
\begin{lstlisting}
def A_func(x):
    return 1/2 + sgn(x)/2 * sqrt(1 - exp(-2*x*x/pi))
\end{lstlisting}

\scalebox{1.5}{
    \begin{adjustbox}{max width=\textwidth, angle=90}

        \begin{tabular}{rrrrrrrrrrrrrrrrrrrrrr}
            \toprule
            Strike & Mid\_Call & Mid\_Put & impl call vol & impl put vol & y     & alpha\_C & R\_C & B\_C & C\_C & A    & alpha\_P & R\_P & B\_P & C\_P & beta\_C & gamma\_C & beta\_P & gamma\_P & CP\_00  & sigma\_C & sigma\_P \\
            \midrule
            2150   & 260.00    & 35.25    & 0.17          & 0.17         & 0.10  & 0.12     & 0.14 & 0.04 & 0.03 & 0.01 & 0.02     & 0.14 & 0.04 & 0.03 & 0.67    & 0.64     & 0.67    & 0.64     & 2135.31 & 0.17     & 0.17     \\
            2175   & 238.85    & 38.95    & 0.17          & 0.17         & 0.09  & 0.11     & 0.13 & 0.04 & 0.03 & 0.00 & 0.02     & 0.13 & 0.04 & 0.03 & 0.71    & 0.53     & 0.72    & 0.53     & 2160.14 & 0.17     & 0.17     \\
            2200   & 218.15    & 43.00    & 0.16          & 0.16         & 0.08  & 0.10     & 0.12 & 0.04 & 0.03 & 0.00 & 0.02     & 0.12 & 0.04 & 0.03 & 0.76    & 0.42     & 0.76    & 0.42     & 2184.97 & 0.16     & 0.16     \\
            2225   & 197.95    & 47.60    & 0.16          & 0.16         & 0.07  & 0.09     & 0.11 & 0.03 & 0.03 & 0.00 & 0.02     & 0.11 & 0.03 & 0.03 & 0.81    & 0.33     & 0.81    & 0.33     & 2209.80 & 0.16     & 0.16     \\
            2250   & 178.15    & 52.65    & 0.15          & 0.15         & 0.05  & 0.08     & 0.10 & 0.03 & 0.03 & 0.00 & 0.02     & 0.10 & 0.03 & 0.03 & 0.86    & 0.24     & 0.86    & 0.24     & 2234.63 & 0.15     & 0.15     \\
            2275   & 159.00    & 58.35    & 0.15          & 0.15         & 0.04  & 0.07     & 0.10 & 0.03 & 0.03 & 0.00 & 0.03     & 0.10 & 0.03 & 0.03 & 0.90    & 0.16     & 0.90    & 0.16     & 2259.46 & 0.15     & 0.15     \\
            2300   & 140.55    & 64.70    & 0.14          & 0.14         & 0.03  & 0.06     & 0.09 & 0.03 & 0.03 & 0.00 & 0.03     & 0.09 & 0.03 & 0.03 & 0.94    & 0.10     & 0.94    & 0.10     & 2284.29 & 0.14     & 0.14     \\
            2325   & 122.85    & 71.85    & 0.14          & 0.14         & 0.02  & 0.05     & 0.08 & 0.03 & 0.03 & 0.00 & 0.03     & 0.08 & 0.03 & 0.03 & 0.97    & 0.05     & 0.97    & 0.05     & 2309.12 & 0.14     & 0.14     \\
            2350   & 106.05    & 79.80    & 0.13          & 0.13         & 0.01  & 0.05     & 0.08 & 0.02 & 0.02 & 0.00 & 0.03     & 0.08 & 0.02 & 0.02 & 0.99    & 0.02     & 0.99    & 0.02     & 2333.95 & 0.13     & 0.13     \\
            2375   & 90.20     & 88.80    & 0.13          & 0.13         & 0.00  & 0.04     & 0.08 & 0.02 & 0.02 & 0.00 & 0.04     & 0.08 & 0.02 & 0.02 & 1.00    & 0.00     & 1.00    & 0.00     & 2358.78 & 0.13     & 0.13     \\
            2400   & 75.45     & 98.80    & 0.12          & 0.12         & -0.01 & 0.03     & 0.07 & 0.02 & 0.02 & 0.00 & 0.04     & 0.07 & 0.02 & 0.02 & 0.99    & 0.01     & 0.99    & 0.01     & 2383.61 & 0.12     & 0.12     \\
            2425   & 61.95     & 110.15   & 0.12          & 0.12         & -0.02 & 0.03     & 0.07 & 0.02 & 0.02 & 0.00 & 0.05     & 0.07 & 0.02 & 0.02 & 0.97    & 0.05     & 0.97    & 0.05     & 2408.44 & 0.12     & 0.12     \\
            2450   & 49.70     & 122.75   & 0.11          & 0.11         & -0.03 & 0.02     & 0.07 & 0.02 & 0.02 & 0.00 & 0.05     & 0.07 & 0.02 & 0.02 & 0.92    & 0.13     & 0.92    & 0.13     & 2433.27 & 0.11     & 0.11     \\
            2475   & 39.05     & 136.85   & 0.11          & 0.11         & -0.04 & 0.02     & 0.07 & 0.02 & 0.01 & 0.00 & 0.06     & 0.07 & 0.02 & 0.01 & 0.85    & 0.25     & 0.85    & 0.25     & 2458.09 & 0.11     & 0.11     \\
            2500   & 29.85     & 152.50   & 0.11          & 0.11         & -0.05 & 0.01     & 0.07 & 0.02 & 0.01 & 0.00 & 0.06     & 0.07 & 0.02 & 0.01 & 0.76    & 0.42     & 0.76    & 0.43     & 2482.92 & 0.11     & 0.11     \\
            2550   & 16.05     & 187.95   & 0.10          & 0.10         & -0.07 & 0.01     & 0.08 & 0.01 & 0.01 & 0.00 & 0.07     & 0.08 & 0.01 & 0.01 & 0.54    & 0.96     & 0.53    & 0.99     & 2532.58 & 0.10     & 0.10     \\
            2600   & 7.95      & 229.45   & 0.09          & 0.09         & -0.09 & 0.00     & 0.09 & 0.01 & 0.00 & 0.00 & 0.09     & 0.09 & 0.01 & 0.00 & 0.32    & 1.79     & 0.30    & 1.87     & 2582.24 & 0.09     & 0.09     \\
            2650   & 3.80      & 275.90   & 0.09          & 0.09         & -0.11 & 0.00     & 0.11 & 0.02 & 0.00 & 0.01 & 0.10     & 0.11 & 0.02 & 0.00 & 0.16    & 2.86     & 0.17    & 2.75     & 2631.90 & 0.09     & 0.09     \\
            2700   & 1.90      & 323.65   & 0.09          & 0.09         & -0.13 & 0.00     & 0.12 & 0.02 & 0.00 & 0.01 & 0.12     & 0.12 & 0.02 & 0.00 & 0.08    & 4.02     & 0.09    & 3.77     & 2681.56 & 0.09     & 0.09     \\
            2800   & 0.65      & 421.60   & 0.10          & 0.10         & -0.16 & 0.00     & 0.15 & 0.03 & 0.00 & 0.01 & 0.15     & 0.15 & 0.03 & 0.00 & 0.02    & 6.00     & 0.03    & 5.51     & 2780.87 & 0.09     & 0.09     \\
            \bottomrule
        \end{tabular}

    \end{adjustbox}
}

The relative difference (compared to the Newtonian method):

\scalebox{0.8}{
    \begin{tabular}{rrrrrrr}
        \toprule
        Strike & impl call vol & impl put vol & sigma\_C & sigma\_P & call vol d [\%] & put vol d [\%] \\
        \midrule
        2150   & 0.172         & 0.172        & 0.170    & 0.170    & -1.375          & -1.371         \\
        2175   & 0.167         & 0.167        & 0.165    & 0.166    & -1.144          & -1.141         \\
        2200   & 0.162         & 0.163        & 0.161    & 0.161    & -0.923          & -0.922         \\
        2225   & 0.158         & 0.158        & 0.157    & 0.157    & -0.716          & -0.716         \\
        2250   & 0.153         & 0.153        & 0.152    & 0.152    & -0.528          & -0.527         \\
        2275   & 0.148         & 0.148        & 0.147    & 0.147    & -0.360          & -0.359         \\
        2300   & 0.143         & 0.143        & 0.143    & 0.143    & -0.217          & -0.217         \\
        2325   & 0.138         & 0.138        & 0.138    & 0.138    & -0.105          & -0.105         \\
        2350   & 0.133         & 0.133        & 0.133    & 0.133    & -0.031          & -0.031         \\
        2375   & 0.129         & 0.129        & 0.129    & 0.129    & -0.002          & -0.002         \\
        2400   & 0.124         & 0.124        & 0.124    & 0.124    & -0.028          & -0.028         \\
        2425   & 0.119         & 0.119        & 0.119    & 0.119    & -0.120          & -0.120         \\
        2450   & 0.115         & 0.115        & 0.114    & 0.114    & -0.292          & -0.293         \\
        2475   & 0.110         & 0.110        & 0.110    & 0.110    & -0.557          & -0.559         \\
        2500   & 0.106         & 0.106        & 0.105    & 0.105    & -0.930          & -0.932         \\
        2550   & 0.099         & 0.098        & 0.097    & 0.096    & -2.019          & -2.063         \\
        2600   & 0.094         & 0.092        & 0.091    & 0.089    & -3.420          & -3.531         \\
        2650   & 0.091         & 0.093        & 0.087    & 0.089    & -4.707          & -4.595         \\
        2700   & 0.091         & 0.094        & 0.086    & 0.089    & -5.478          & -5.360         \\
        2800   & 0.096         & 0.100        & 0.091    & 0.094    & -5.745          & -5.771         \\
        \bottomrule
    \end{tabular}
}

Here, implied means the Newtonian method, and sigma refers to the explicit formula.
Close to the spot price of the index the error is remarkably small.

\section{exercise}
\subsection*{i}
\scalebox{0.8}{
    \begin{tabular}{lrrrrrrrr}
        \toprule
                   & BAC   & BCS    & CS     & GS    & JPM    & MS     & RBS    & UBS    \\
        \midrule
        2016-09-16 & -1.59 & -6.23  & -4.66  & -1.52 & -1.25  & 0.00   & -10.33 & -6.20  \\
        2016-09-09 & -1.62 & -1.72  & 3.47   & -0.36 & -1.24  & -0.94  & -0.73  & -0.20  \\
        2016-09-02 & 1.80  & 7.51   & 7.45   & 2.33  & 1.92   & 2.34   & 5.61   & 4.85   \\
        2016-08-26 & 3.75  & 2.61   & 3.52   & -0.16 & 0.55   & 2.00   & 5.08   & 4.23   \\
        2016-08-19 & 2.08  & -0.35  & -0.83  & 1.83  & 0.83   & 4.73   & -3.34  & -2.89  \\
        2016-08-12 & -0.93 & 5.36   & 6.37   & 0.72  & -1.48  & 0.55   & 6.93   & 2.97   \\
        2016-08-05 & 3.86  & -1.82  & -2.16  & 2.07  & 3.64   & 0.97   & -6.67  & -2.32  \\
        2016-07-29 & 0.76  & 2.74   & 0.17   & -1.00 & -0.11  & 0.03   & 2.41   & 3.69   \\
        2016-07-22 & 5.27  & 0.25   & 1.85   & -0.76 & -0.22  & 3.25   & 1.22   & 0.91   \\
        2016-07-15 & 3.72  & 9.29   & 8.32   & 7.49  & 3.80   & 6.22   & 11.82  & 7.51   \\
        2016-07-08 & 0.53  & -3.56  & -3.77  & 1.44  & 0.93   & 1.74   & -4.56  & -6.99  \\
        2016-07-01 & 0.77  & -14.62 & -5.56  & 4.50  & 3.59   & 5.71   & -15.10 & -5.59  \\
        2016-06-24 & -2.99 & -7.59  & -8.29  & -2.60 & -4.30  & -3.12  & -16.20 & -5.17  \\
        2016-06-17 & -3.11 & 0.63   & 0.97   & -2.84 & -2.44  & -0.90  & 6.93   & 2.65   \\
        2016-06-10 & -4.09 & -9.04  & -6.89  & -3.71 & -1.24  & -3.77  & -12.93 & -3.18  \\
        2016-06-03 & -2.76 & -3.67  & -6.64  & -2.42 & -1.21  & -3.60  & -5.43  & -4.15  \\
        2016-05-27 & 2.48  & 7.38   & 4.38   & 3.67  & 3.02   & 2.84   & 9.04   & 4.11   \\
        2016-05-20 & 4.61  & 8.20   & 6.61   & -0.53 & 3.77   & 3.40   & 11.39  & 0.75   \\
        2016-05-13 & -1.63 & -0.21  & -3.16  & -2.21 & -0.65  & -1.37  & -2.73  & -0.12  \\
        2016-05-06 & -3.09 & -6.37  & -7.96  & -3.21 & -2.53  & -2.99  & -7.98  & -9.67  \\
        2016-04-29 & -3.64 & 1.41   & -2.62  & -1.58 & -1.20  & -1.78  & -7.26  & 3.97   \\
        2016-04-22 & 7.93  & 4.76   & 4.90   & 5.19  & 3.39   & 7.53   & 9.45   & 4.01   \\
        2016-04-15 & 8.70  & 11.69  & 9.73   & 5.48  & 7.15   & 8.46   & 11.73  & 5.27   \\
        2016-04-08 & -5.01 & -1.85  & -3.49  & -5.97 & -2.84  & -6.97  & -4.48  & -4.05  \\
        2016-04-01 & -0.88 & -1.37  & -2.09  & 4.46  & 0.66   & 2.41   & -1.88  & -2.65  \\
        2016-03-24 & -0.80 & -7.51  & -5.03  & -2.92 & -1.65  & -5.14  & -7.01  & -2.87  \\
        2016-03-18 & 0.00  & -2.07  & -5.08  & 2.38  & 1.92   & 1.08   & 2.24   & -1.47  \\
        2016-03-11 & 1.85  & 0.50   & 3.17   & -1.85 & -1.18  & -0.50  & 1.82   & 2.79   \\
        2016-03-04 & 7.02  & 5.37   & 13.11  & 4.84  & 4.36   & 3.86   & 4.44   & 7.91   \\
        2016-02-26 & 4.70  & 0.98   & 3.57   & 2.27  & -0.48  & 4.83   & -10.51 & 2.82   \\
        2016-02-19 & 1.51  & 0.22   & -4.63  & 0.53  & 0.57   & 3.94   & 1.15   & -1.20  \\
        2016-02-12 & -7.72 & -8.55  & -7.74  & -6.61 & -0.45  & -5.17  & -0.57  & -1.25  \\
        2016-02-05 & -8.42 & -6.68  & -15.84 & -3.15 & -2.94  & -5.91  & -5.02  & -8.35  \\
        2016-01-29 & 4.28  & -1.91  & -0.22  & 3.00  & 4.48   & 1.65   & -2.38  & -0.12  \\
        2016-01-22 & -6.22 & 0.27   & -4.60  & 0.80  & -0.16  & -1.39  & 0.27   & 0.24   \\
        2016-01-15 & -4.87 & -6.16  & -4.49  & -5.08 & -3.19  & -8.49  & -7.72  & -4.32  \\
        2016-01-08 & -9.69 & -9.88  & -9.73  & -9.04 & -10.17 & -10.78 & -8.00  & -10.33 \\
        2015-12-31 & -2.55 & -2.92  & -1.86  & -1.23 & -0.86  & -2.06  & -2.63  & -1.58  \\
        2015-12-24 & 3.04  & 3.73   & 3.80   & 3.98  & 3.42   & 3.80   & 4.95   & 2.88   \\
        2015-12-18 & 0.18  & 0.39   & 3.80   & -0.61 & 0.52   & -2.46  & 0.58   & 4.25   \\
        2015-12-11 & -6.01 & -8.03  & -7.30  & -7.07 & -5.63  & -9.17  & -6.50  & -6.57  \\
        2015-12-04 & 2.12  & 3.64   & 2.82   & 0.09  & 1.07   & 4.56   & 0.44   & 2.40   \\
        2015-11-27 & -0.96 & 0.15   & -2.05  & -0.52 & -0.55  & -0.38  & -1.61  & -1.54  \\
        2015-11-20 & 2.62  & -1.32  & -2.77  & 0.57  & 3.02   & -0.06  & 0.32   & 2.69   \\
        2015-11-13 & -4.18 & -3.27  & -5.87  & -4.41 & -4.24  & -4.18  & -3.42  & -3.90  \\
        2015-11-06 & NaN   & NaN    & NaN    & NaN   & NaN    & NaN    & NaN    & NaN    \\
        \bottomrule
    \end{tabular}
}

The last row with "not a number" values means that
there the returns cannot be calculated for that week,
because there is no data for previous week.

We now aim to solve
\[
    y = X\beta + \epsilon
\]

Where:
\begin{itemize}
    \item \( y \) is the vector of the dependent variable, JPM's data (weekly return or value).
    \item \( X \) is the matrix of independent variables, with an intercept term included (a column of ones is added to \( X \)). All the stocks or just some of them.
    \item \( \beta \) is the vector of coefficients, including the intercept.
\end{itemize}

To find the coefficients \( \beta \) that minimize the residuals, we use the OLS formula:
\[
    \beta = (X^T X)^{-1} X^T y
\]
\subsection*{ii}
Here, the data will be the all returns.

\begin{itemize}
    \item JPM's returns \( y \) is:
          \[
              y = \begin{bmatrix}
                  -1.24  \\
                  -1.24  \\
                  1.92   \\
                  0.55   \\
                  0.83   \\
                  \vdots \\
              \end{bmatrix}
          \]

    \item The independent variables \( X \), including a column of ones for the intercept, look like this:
          \[
              X = \begin{bmatrix}
                  1      & -1.59  & -6.23  & -4.66  & -1.52  & 0.00   & -10.33 & -6.20  \\
                  1      & -1.62  & -1.72  & 3.47   & -0.36  & -0.94  & -0.73  & -0.20  \\
                  1      & 1.80   & 7.51   & 7.45   & 2.33   & 2.34   & 5.61   & 4.85   \\
                  1      & 3.75   & 2.61   & 3.52   & -0.16  & 2.00   & 5.08   & 4.23   \\
                  1      & 2.08   & -0.35  & -0.83  & 1.83   & 4.73   & -3.34  & -2.89  \\
                  \vdots & \vdots & \vdots & \vdots & \vdots & \vdots & \vdots & \vdots \\
              \end{bmatrix}
          \]

    \item The coefficients vector \( \beta \) is calculated but not yet known:
          \[
              \beta = \begin{bmatrix}
                  \beta_0 \\
                  \beta_1 \\
                  \beta_2 \\
                  \beta_3 \\
                  \beta_4 \\
                  \beta_5 \\
                  \beta_6 \\
                  \beta_7 \\
              \end{bmatrix}
          \]
\end{itemize}

The result is:

$$\displaystyle \beta = (X^T X)^{-1} X^T y = \left[\begin{matrix}0.193\\0.338\\-0.112\\-0.153\\0.472\\-0.003\\0.112\\0.154\end{matrix}\right]$$

From here, the predicted values and error:
\[{y_{{\rm{pred}}}} = X\beta \qquad \left\| {{y_{{\rm{pred}}}} - y} \right\| = {\rm{error}} = 8.76\]

\subsection*{iii}
Here we fit only to GS, MS, and BAC.

$$\displaystyle X  = \left[\begin{matrix}1.0 & -1.52 & 0.0 & -1.59\\1.0 & -0.36 & -0.94 & -1.62\\1.0 & 2.33 & 2.34 & 1.8\\1.0 & -0.16 & 2.0 & 3.75\\ \vdots & \vdots & \vdots & \vdots\end{matrix}\right]$$

The results are:

$$\displaystyle \beta = \left[\begin{matrix}0.209\\0.479\\0.015\\0.258\end{matrix}\right]\qquad \left\| {{y_{{\rm{pred}}}} - y} \right\| = {\rm{error}} = 9.52$$
\subsection*{iv}
Instead the returns, use the stock values itself. Use JPM as $y$, add a new column with ones for intercept, like in ii. The results are

$$\displaystyle \beta = \left[\begin{matrix}20.168\\1.948\\1.414\\-1.648\\0.123\\0.09\\-0.057\\0.126\end{matrix}\right] \qquad \left\|  {{y_{{\rm{pred}}}} - y} \right\| = {\rm{error}} = 5.40$$

This is smaller than ii.
Consider a stock A that goes up by 1 every week.
Consider another stock B that goes up by 2 on even weeks,
and goes down by 1 on odd weeks.
You can explain the cumulative changes of A
by the cumulative changes in B (or vice versa) very well,
but the explanation of the weekly returns (with a linear model) will be much worse.
\section{exercise}
\subsection*{i}
\textbf{Parameters:}
\begin{itemize}
    \item Stock A: \$80 per share, 150 shares.
    \item Stock B: \$65 per share, 500 shares.
\end{itemize}

\textbf{Total Investment Value:}
\begin{align*}
    V_1              & = 80 \times 150 = \$12000 \quad (\text{Stock A}) \\
    V_2              & = 65 \times 500 = \$32500 \quad (\text{Stock B}) \\
    V_{\text{total}} & = V_1 + V_2 = \$44500
\end{align*}

\textbf{Portfolio Weights:}
\begin{align*}
    w_1 & = \frac{V_1}{V_{\text{total}}} \approx 0.2697 \quad (\text{26.97\%}) \\
    w_2 & = \frac{V_2}{V_{\text{total}}} \approx 0.7303 \quad (\text{73.03\%})
\end{align*}
\subsection*{ii}
\begin{align*}
    w_1 + w_2         & = 1 \quad \text{(Sum of weights)}   \\
    0.10w_1 + 0.05w_2 & = 0.08 \quad \text{(Target return)}
\end{align*}

Solving the above equations, substitute \( w_2 = 1 - w_1 \) into the return equation:

\begin{align*}
    0.10w_1 + 0.05(1 - w_1) & = 0.08                 \\
    0.05w_1                 & = 0.03                 \\
    w_1                     & = 0.6, \quad w_2 = 0.4
\end{align*}
Portfolio weights should be 60\% in Stock 1 and 40\% in Stock 2.

\subsection*{iii}

Let:
\begin{itemize}
    \item \(\mathbf{w} = \begin{bmatrix} w_1 \\ w_2 \end{bmatrix}\) be the weights of the assets in the portfolio.
    \item \(\mathbf{r} = \begin{bmatrix} r_1 \\ r_2 \end{bmatrix}\) be the expected returns, where \(r_1 = 10\%\), \(r_2 = 5\%\).
    \item \(\Sigma = \begin{bmatrix} \sigma_1^2 & \rho \sigma_1 \sigma_2 \\ \rho \sigma_1 \sigma_2 & \sigma_2^2 \end{bmatrix}\) be the covariance matrix, where \(\sigma_1 = 35\%\), \(\sigma_2 = 15\%\), and \(\rho = 15\%\).
\end{itemize}

The variance of the portfolio, \(\sigma_p^2\), is given by:
\[
    \sigma_p^2 = \mathbf{w}^\top \Sigma \mathbf{w} = w_1^2 \sigma_1^2 + 2 w_1 w_2 \rho \sigma_1 \sigma_2 + w_2^2 \sigma_2^2\qquad w_1 + w_2 = 1
\]

The standard deviation, \(\sigma_p\), is the square root of the variance:
\[
    \sigma_p = \sqrt{\mathbf{w}^\top \Sigma \mathbf{w}}
\]

\[
    \sigma_p = \sqrt{w_1^2 \sigma_1^2 + 2 w_1 (1 - w_1) \rho \sigma_1 \sigma_2 + (1 - w_1)^2 \sigma_2^2}
\]
and the expected return of the portfolio, \(r_p\), is given by:
\[
    r_p = \mathbf{w}^\top \mathbf{r} = w_1 r_1 + w_2 r_2
\]

The solutions are:
\[
    w_1 = -0.42, \quad w_2 = 1.42\qquad r_p = 0.0289
\]
and
\[
    w_1 = 0.646, \quad w_2 = 0.354\qquad r_p = 0.0823
\]
\section{exercise}
The proportional weights of each asset in a tangency portfolio, excluding cash, are calculated as:
\begin{align*}
    \text{Weight of the first asset}  & = \frac{25}{85} \approx 29.41\% \\
    \text{Weight of the second asset} & = \frac{15}{85} \approx 17.65\% \\
    \text{Weight of the third asset}  & = \frac{45}{85} \approx 52.94\%
\end{align*}

\section{exercise}
\subsection*{i}
$$\displaystyle \Sigma = \left[\begin{matrix}0.022 & 0.011\\0.011 & 0.062\end{matrix}\right]\qquad \Sigma^{-1} = \displaystyle \left[\begin{matrix}48.84 & -8.791\\-8.791 & 17.582\end{matrix}\right]$$
$$\bar\mu = \mu - r_f = \displaystyle \left[\begin{matrix}0.06\\0.12\end{matrix}\right] - 0.2= \displaystyle \left[\begin{matrix}0.04\\0.1\end{matrix}\right]$$
Weights for the tangency portfolio are computed as:
\[
    w = \frac{\Sigma^{-1} \bar{\mu}}{\mathbf{1}^T \Sigma^{-1} \bar{\mu}} = \displaystyle \left[\begin{matrix}0.433\\0.567\end{matrix}\right]
\]

Multiply this by \$20MM to get the allocation: \$8.66MM to asset 1 and \$11.35MM to asset 2.

\subsection*{ii}
To calculate the specific allocations to cash, Asset 1, and Asset 2:
\begin{itemize}
    \item Let \( k \) be the scaling factor for the weights from the tangency portfolio to achieve the target return of $R_\text{target} = 8\%$.
    \item The weights of the assets are scaled as:
          \[ w' = k \cdot w_{\text{tangency}} \]
    \item The allocation to cash, which ensures that the total of all allocations sums to 100\%, is given by:
          \[ w_{\text{cash}} = 1 - (w'_1 + w'_2) \]
    \item Where \( w'_1 \) and \( w'_2 \) are the scaled weights for Asset 1 and Asset 2, respectively.
\end{itemize}

The scaling factor \( k \) is calculated as:
\[ k = \frac{(R_{\text{target}} - r_f)}{w_{\text{tangency}}^T \bar{\mu}} \]

The results:

$$k=0.810\qquad w' = \displaystyle \left[\begin{matrix}0.351\\0.46\end{matrix}\right] \qquad w_{\text{cash}} = 0.189  $$
It means that weight for asset 1 is
$35.1\%$, $46.0\%$ for asset 2,
and the remaining $18.9\%$ is for cash.
Given \$20MM for the portfolio: \$7.02MM to asset 1, \$9.19MM to asset 2, and \$3.79MM stays is cash.
Given the scaled weights \( w' \) for the portfolio, and the covariance matrix \( \Sigma \) of the assets, the standard deviation of the portfolio, \( \sigma_p \), is calculated using the formula:
\[
    \sigma_p = \sqrt{w'^T \Sigma w'} = 14.00\%
\]
Multiply it by \$20MM to get it in dollars.
\subsection*{iii}
Allocation weights and amounts for the 2 assets, in order:
$$\displaystyle w' = \left[\begin{matrix}0.761\\0.996\end{matrix}\right] \Rightarrow \displaystyle \left[\begin{matrix}15.22\\19.92\end{matrix}\right]\$\text{MM}$$
And $w_{\text{cash}} = -0.756 \Rightarrow \$-15.12\text{MM}$, and $\sigma_p = 30.34\%$.
Multiply it by \$20MM to get it in dollars.

\subsection*{iv}
Given the weights of the tangency portfolio \( w_{\text{tangency}} \) and its covariance matrix \( \Sigma \), to achieve a portfolio with a standard deviation of 20\%, we scale the weights by a factor \( k \), where \( k \) is determined as follows:

The standard deviation of the scaled portfolio \( \sigma_{\text{scaled}} \) is given by:
\[ \sigma_{\text{scaled}} = k \sqrt{w_{\text{tangency}}^T \Sigma w_{\text{tangency}}} \]
Setting \( \sigma_{\text{scaled}} = 0.20 \), we solve for \( k \):
\[ k = \frac{0.20}{\sqrt{w_{\text{tangency}}^T \Sigma w_{\text{tangency}}}} \]

The new weights \( w_{\text{scaled}} \) are then:
\[ w_{\text{scaled}} = k \times w_{\text{tangency}} \]

The expected return of this portfolio is:
\[ E(R_{\text{scaled}}) = w_{\text{scaled}}^T E(R) \]

$$\displaystyle w_\text{scaled} = \left[\begin{matrix}0.501\\0.656\end{matrix}\right]\qquad w_\text{cash} = -0.157$$
That is, by multiplying with \$20MM, \$10.03MM to asset 1, \$13.13MM to asset 2, and \$3.16MM cash loan.

The expected return on the scaled tangency portfolio is 0.1088, but it is negative on the cash.
The net expected return of the portfolio,
\( E(R_{\text{net}}) \), is calculated by considering both the returns from the assets and the cost of borrowing:

\[ E(R_{\text{net}}) = w_{\text{scaled}}^T E(R) + w_{\text{cash}} r_f = 0.1057 \Rightarrow \$2.11\text{MM}\]

where:
\begin{itemize}
    \item \( w_{\text{scaled}} \) are the scaled weights of the assets,
          contributing positively to the portfolio's return.
    \item \( E(R) \) is the vector of expected returns for each asset.
    \item \( w_{\text{cash}} \) is the negative weight representing the borrowed funds;
          it is negative and reflects the cost of borrowing.
    \item \( r_f =0.02\) is the borrowing rate, indicating the cost per unit of negative cash held.
    \item \( w_{\text{cash}} r_f \) quantifies the borrowing cost,
          effectively reducing the overall expected return by the interest expense on the borrowed amount.
\end{itemize}
\subsection*{v}
$$\displaystyle w_\text{scaled} = \left[\begin{matrix}0.752\\0.985\end{matrix}\right] \Rightarrow \left[\begin{matrix}15.044\\19.694\end{matrix}\right] \text{\$MM}$$
$$w_\text{cash} = -0.7369 \Rightarrow -14.74 \$\text{MM}\qquad E(R_{\text{net}}) = 0.149 \Rightarrow \$2.97\text{MM}$$

\subsection*{vi}
Modifying the risk free interest rate modifies the $\bar\mu$, so the tangency portfolio needs to be recalcualted
$$\displaystyle w = \left[\begin{matrix}0.391\\0.609\end{matrix}\right]$$

\subsubsection*{vi/i}
$$\displaystyle w_\text{scaled} = \left[\begin{matrix}0.3\\0.468\end{matrix}\right]\Rightarrow \displaystyle \left[\begin{matrix}6.009\\9.365\end{matrix}\right] \$\text{MM}$$
For cash,
we have
$w_\text{cash} = 0.231 \Rightarrow \$4.62\text{MM}$.
    This says that to achieve the same return with minimal risk,
    we can leave more money in the treasury. While amount in asset 1 is decreased, asset 2 is increased.

    \subsubsection*{vi/ii}
    $$\displaystyle w_\text{scaled} = \left[\begin{matrix}0.437\\0.681\end{matrix}\right] \Rightarrow \displaystyle \left[\begin{matrix}8.741\\13.624\end{matrix}\right] \$\text{MM}$$
    For cash, $w_\text{cash} = -0.118 \Rightarrow \$2.37\text{MM}$.
The trend is similar, higher value for cash allocatin (i.e. smaller loan),
lower amount in asset 1, and higher amount in asset 2.

\section{exercise}
\subsection*{i}
$$\displaystyle \Sigma = \left[\begin{matrix}0.0256 & -0.016 & -0.006\\-0.016 & 0.04 & -0.0125\\-0.006 & -0.0125 & 0.0625\end{matrix}\right] \Rightarrow \Sigma ^{-1} = \displaystyle \left[\begin{matrix}58.3603 & 26.7679 & 10.9562\\26.7679 & 38.9442 & 10.3586\\10.9562 & 10.3586 & 19.1235\end{matrix}\right]$$
$$\displaystyle \bar\mu = \mu - r_f = \displaystyle \left[\begin{matrix}0.05\\0.08\\0.11\end{matrix}\right] - 0.02 = \left[\begin{matrix}0.03\\0.06\\0.09\end{matrix}\right]$$
$$\displaystyle w_\text{tan} = \frac{\Sigma^{-1} \bar\mu}{\mathbf{1}^T \Sigma^{-1} \bar\mu} = \left[\begin{matrix}0.392\\0.367\\0.241\end{matrix}\right]$$
The expected return: $\mu_{\text{tan}} = w_{\text{tan}}^T \mu = 0.07547$
$$\sigma_{\text{tan}} = \sqrt{w_{\text{tan}}^T \Sigma w_{\text{tan}}} = 0.07073$$

\subsection*{ii}
We want expected return of 10\%, which is higher than the tangency portfolio's return.
$$\displaystyle w_\text{scaled} = \left[\begin{matrix}0.519\\0.487\\0.319\end{matrix}\right],$$
with $w_\text{cash} = -0.3249$ and $\sigma_\text{scaled} = 0.0937$.
\subsection*{iii}
We want std of 20\%, which is higher than the tangency portfolio's std.
$$\displaystyle w_\text{scaled} =\displaystyle \left[\begin{matrix}1.108\\1.038\\0.681\end{matrix}\right]$$
with $w_\text{cash} =-1.83$, and $E[R_{\text{net}}] = 0.213$.

\section{exercise}
\subsection*{i}
$$ \Sigma = \left[\begin{matrix}0.09 & -0.01 & -0.03 & -0.02\\-0.01 & 0.063 & 0.02 & -0.01\\-0.03 & 0.02 & 0.1225 & -0.015\\-0.02 & -0.01 & -0.015 & 0.0576\end{matrix}\right]$$
$$\Sigma^{-1} = \displaystyle \left[\begin{matrix}14.0 & 2.0 & 3.86 & 6.22\\2.0 & 17.46 & -1.97 & 3.21\\3.86 & -1.97 & 9.87 & 3.57\\6.22 & 3.21 & 3.57 & 21\end{matrix}\right]\qquad\bar\mu =  \left[\begin{matrix}0.036\\0.03\\0.053\\0.027\end{matrix}\right]$$
From these,
$$\displaystyle w_\text{tan} = \frac{\Sigma^{-1} \bar{\mu}}{\mathbf{1}^T \Sigma^{-1} \bar{\mu}} = \left[\begin{matrix}0.285\\0.176\\0.213\\0.327\end{matrix}\right]$$
$$E[R_p] = w^T \mu + r_f =  0.0506$$
\[
    \sigma_p = \sqrt{w^T \Sigma w}  = 0.10403
\]
\[
    \text{Sharpe Ratio} = \frac{E[R_p] - r_f}{\sigma_p} = 0.3423
\]
\subsection*{ii}
We aim for 5\% return, which is slightly lower than the expected return from the tangency portfolio,
i.e. we will leave some cash in the treasury. For the scaling:
\[
    k = \alpha = \frac{R_{\text{target}} - r_f}{E[R_T] - r_f}
\]
$$\displaystyle w_\text{scaled} = \left[\begin{matrix}0.28\\0.173\\0.209\\0.321\end{matrix}\right]$$
with $w_\text{cash} =  0.017$.
\[
    \sigma_{\text{new}} = \alpha \sigma_p = 0.10223
\]
It is indeed smaller than the $\sigma$ of the tangency portfolio.
\[
    SR_{\text{new}} = \frac{R_{\text{target}} - r_f}{\sigma_{\text{new}}} = 0.3423
\]
Which is the same as before, because
The new Sharpe Ratio ($SR_{\text{new}}$) can be expressed as:
\[
    SR_{\text{new}} = \frac{E[R_{\text{new}}] - r_f}{\sigma_{\text{new}}}
\]
\[
    SR_{\text{new}} = \frac{\alpha E[R_T] + (1-\alpha) r_f - r_f}{\alpha \sigma_T}
\]
\[
    SR_{\text{new}} = \frac{\alpha (E[R_T] - r_f)}{\alpha \sigma_T}
\]
\[
    SR_{\text{new}} = \frac{E[R_T] - r_f}{\sigma_T}
\]
\[
    SR_{\text{new}} = SR_T
\]
\subsection*{iii}
The 29\% standard deviation is much higher than the tangency portfolio's, so we expect a high leverage.
$$\displaystyle w_\text{scaled} = \left[\begin{matrix}0.793\\0.49\\0.592\\0.912\end{matrix}\right]$$
with $w_\text{cash} = -1.788$ and $\text{net expected return} = 0.1142$, and $\text{Sharpe Ratio} = \frac{E[R_p] - r_f}{\sigma_p} = 0.3423$.

\subsection*{iv}
Derive the formula for the weights. The weights must sum to one:
\[
    \sum_{i=1}^n w_i = 1
\]
Using a Lagrange multiplier \(\lambda\):
\[
    L(w, \lambda) = w^T \Sigma w + \lambda (1 - w^T \mathbf{1})
\]
Setting the derivative of the Lagrangian with respect to \(w\) and \(\lambda\) to zero, we get:
\[
    \frac{\partial L}{\partial w} = 2 \Sigma w - \lambda \mathbf{1} = 0
\]
\[
    \frac{\partial L}{\partial \lambda} = 1 - w^T \mathbf{1} = 0
\]

From the first equation:
\[
    w = \frac{1}{2} \Sigma^{-1} \lambda \mathbf{1}
\]
Using the second condition to find \(\lambda\):
\[
    1 = \mathbf{1}^T w = \frac{1}{2} \lambda \mathbf{1}^T \Sigma^{-1} \mathbf{1}
\]
\[
    \lambda = \frac{2}{\mathbf{1}^T \Sigma^{-1} \mathbf{1}}
\]
Substituting back for \(w\):
\[
    w = \frac{\Sigma^{-1} \mathbf{1}}{\mathbf{1}^T \Sigma^{-1} \mathbf{1}} = \displaystyle \left[\begin{matrix}0.271\\0.215\\0.16\\0.354\end{matrix}\right]
\]
\[
    \sigma_p = \sqrt{w^T \Sigma w} = 0.102,
\]
The sharpe ratio is then 0.3356, which is indeed lower than the tangency portfolio's. Note that the standard deviations for the different assets are $\sigma_X = \displaystyle \left[\begin{matrix}0.3\\0.25\\0.35\\0.24\end{matrix}\right]$
\section{exercise}
\subsection*{i}
Given the problem to show that the asset allocation for a minimum variance portfolio with an expected return \(\mu_P\) can be expressed in terms of the asset weights vector \(w_T\) of the tangency portfolio.

The minimum variance portfolio that achieves a certain expected return, \(\mu_P\), can be composed of cash (risk-free asset) and the tangency portfolio. We denote:
\begin{itemize}
    \item \(w_{\text{min}}\) as the weight vector of the minimum variance portfolio.
    \item \(w_T\) as the weight vector of the tangency portfolio.
    \item \(r_f\) as the risk-free rate.
\end{itemize}
The expected return on the portfolio \(w_{\text{min}}\) is a linear combination of the risk-free asset and the tangency portfolio, scaled by some factor \(\alpha\). The expected return \(\mu_P\) can be expressed as:
\[
    \mu_P = \alpha \mu_T + (1-\alpha) r_f
\]
where \(\mu_T\) is the expected return of the tangency portfolio.

Rearrange the equation to solve for \(\alpha\):
\[
    \alpha (\mu_T - r_f) = \mu_P - r_f \quad \Rightarrow \quad \alpha = \frac{\mu_P - r_f}{\mu_T - r_f}
\]

The weights \(w_{\text{min}}\) are a scaled version of \(w_T\), since the entire risky part of the portfolio is just a scaled tangency portfolio:
\[
    w_{\text{min}} = \alpha w_T \quad \Rightarrow \quad w_{\text{min}} = \frac{\mu_P - r_f}{\mu_T - r_f} w_T
\]
Assuming \(\mu_T\) is typically defined or can be simplified as \(\mu_T' w_T\), where \(\mu_T' = w_T^T \mu\) and \(\mu\) is the vector of expected returns of the assets. This aligns with the provided formula.

\subsection*{ii}

\begin{myverb}
    Input:
    w_T   // Vector of weights for the tangency portfolio
    r_f   // Risk-free rate
    mu    // Vector of expected returns
    mu_P  // Desired expected return

    Output:
    w_min // Vector of weights for the min var portf

    1. Start
    2. Compute the expected return of the tangency portfolio:
    mu_T = w_T^T * mu   // Dot product

    3. Compute the scaling factor alpha:
    alpha = (mu_P - r_f) / (mu_T - r_f)

    4. Compute the asset weights for the
    minimum variance portfolio:
    w_min = alpha * w_T                  // Scale the tan portf
    5. Return w_min
\end{myverb}

In case the tangency portfolio is not given,
run the subroutine at the end of the doc to get it.

\section{exercise}
\subsection*{i}
From risky assets and risk-free cash calculate the weights of the portfolio, including cash, given:
\begin{itemize}
    \item Tangency portfolio weights, \( w_T \), that maximize the Sharpe ratio.
    \item Covariance matrix of asset returns, \( \Sigma \).
    \item Target variance of the portfolio, \( \sigma_P^2 \).
    \item Cash with zero variance and return \( r_f \).
\end{itemize}

The total portfolio \( w_{\text{max}} \) can be described as a combination of the tangency portfolio and cash:
\[ w_{\text{extrm}} = \alpha w_T + (1 - \alpha) \cdot \text{cash},\]
where \( \alpha \) is a scalar determining the amount of money invested in the tangency portfolio.
With this formula, we can only ensure that the portfolio is extremal,
using the Lagrange multiplicator technique,
and the maximum will be choosen later.
Since cash has zero variance,
the variance of the entire portfolio is contributed only by the tangency portfolio:
\[ \text{Var}(w_{\text{extrm}}) = \alpha^2 w_T^T \Sigma w_T \]
To meet the specific target variance \( \sigma_P^2 \), we set:
\[ \alpha^2 w_T^T \Sigma w_T = \sigma_P^2 \]
Solving for \( \alpha \), we get:
\[ \alpha = \frac{\sigma_P}{\sqrt{w_T^T \Sigma w_T}} \]
therefore
\[ w_{\text{extrm}} = \alpha w_T = \frac{\sigma_P}{\sqrt{w_T^T \Sigma w_T}} w_T\]
and
\[ w_{\text{cash}} = 1 - \mathbf{1}^T w_{\text{extrm}}.\]

This ensures that the variance is as desired, and the return is extremal.
In the case when the portfolio underperforms the risk-free cash,
then short the assets and keep cash, this is represented by the sign operator.
What the argument contains is actually just proportional to the tangency portfolio weights,
not the weights themself,
but the SGN operator is insensitive to a constant positive normalization factor.
So the final result is
\[ w_{\text{max}} = \sgn\left(w_T^T\bar\mu\right) \frac{\sigma_P}{\sqrt{w_T^T \Sigma w_T}} w_T = \sgn\left(\mathbf{1}^T\Sigma^{-1}\bar\mu\right) \frac{\sigma_P}{\sqrt{w_T^T \Sigma w_T}} w_T\]

Note that once we have the tangency portfolio, we don't need the inverse of the covariance matrix anymore, so this doesn't need to (re)calculated or used.

\subsection*{ii}
\begin{myverb}
    # Inputs:
    # w_T: Vector of weights for the tangency portfolio
    # bar_mu: Vector of expected excess returns over the risk-free rate
    # Sigma: Covariance matrix of the asset returns
    # sigma_P: Target standard deviation of the portfolio

    # Outputs:
    # w_max: Vector of weights for the adjusted tangency portfolio
    # w_cash: the cash allocation in the portfolio

    # Step 1: Calculate the current variance of the tangency portfolio
    current_variance = dot(w_T, dot(Sigma, w_T))  # dot is the matrix multiplication

    # Step 2: Calculate the sign of the dot product of the tangency portfolio weights
    # and the expected excess returns vector
    sign_factor = sign(dot(w_T, bar_mu))  # signum function determining the direction

    # Step 3: Calculate the scalar alpha to match the target variance, adjusted by the sign
    alpha = sign_factor * sigma_P / sqrt(current_variance)

    # Step 4: Calculate the new weights for the tangency portfolio
    w_max = alpha * w_T  # Element-wise multiplication by scalar

    # Step 5: Calculate the cash allocation
    w_cash = 1 - sum(w_max)  # sum of all elements in w_max to ensure total is 1

    # Output the results
    return (w_max, w_cash)
\end{myverb}

In case the tangency portfolio is not given,
run this subroutine to get it:
\begin{myverb}
    # Inputs:
    # R: Vector of expected returns for each asset
    # rf: Risk-free interest rate
    # Sigma: Covariance matrix of asset returns

    # Outputs:
    # w_T: Vector of weights for the tangency portfolio

    # Step 1: Calculate excess returns over the risk-free rate
    bar_mu = R - rf  # Subtract rf from each asset's expected return

    # Step 2: Calculate the inverse of the covariance matrix
    Sigma_inv = inverse(Sigma)  # Compute the inverse of the covariance matrix

    # Step 3: Compute the weights of the tangency portfolio
    # Formula: w_T = (Sigma_inv * bar_mu) / (1^T * Sigma_inv * bar_mu)
    numerator = dot(Sigma_inv, bar_mu)  # Inverse covariance * excess returns
    denominator = dot(ones(numerator.shape[0]), numerator)  # Vector of ones * numerator

    w_T = numerator / denominator  # Element-wise division for portfolio weights

    # Step 4:
    return w_T
\end{myverb}
\end{document}