\documentclass{article}
\usepackage{amsmath}
\usepackage{amssymb}
\usepackage{array}
\usepackage{graphicx} % Required for \scalebox
\usepackage{hyperref}
\usepackage{listings}
\usepackage{booktabs}
\usepackage{xcolor}
\usepackage{graphicx}
\usepackage{verbatim}
\usepackage[extreme]{savetrees} % tighter margins
\usepackage{lmodern} % Use scalable Latin Modern fonts
\title{Baruch SIP NLA Quiz 1}
\author{Daniel Tuzes}

\begin{document}
\maketitle

\section{exercise 1}
\begin{equation}
    D^{-1}=
    \left[\begin{matrix}1 & 0 & 0\\0 & 2/3 & 0\\0 & 0 & 0.4\end{matrix}\right]
\end{equation}

\begin{equation}
    \Omega = D^{-1}\cdot \Sigma D^{-1} = \left[\begin{matrix}1 & -0.35 & 0.15\\-0.35 & 1.0 & 0.05\\0.15 & 0.05 & 1.0\end{matrix}\right]
\end{equation}

\section{exercise 2}
\subsection{a}
Daily percentage returns (to be divided by 100 to get values in the unit of 1):

\begin{tabular}{lrrr}
    \toprule
    Date       & Dow Jones \% Return & NASDAQ \% Return & S\&P 500 \% Return \\
    \midrule
    2012-07-26 & NaN                 & NaN              & NaN                \\
    2012-07-27 & 1.456634            & 2.241078         & 1.908060           \\
    2012-07-30 & -0.020267           & -0.414119        & -0.048342          \\
    2012-07-31 & -0.492083           & -0.214540        & -0.431675          \\
    2012-08-01 & -0.250218           & -0.656910        & -0.289998          \\
    2012-08-02 & -0.749453           & -0.357509        & -0.750371          \\
    2012-08-03 & 1.687181            & 1.997752         & 1.904029           \\
    2012-08-06 & 0.162948            & 0.741602         & 0.232928           \\
    2012-08-07 & 0.389479            & 0.867919         & 0.510676           \\
    2012-08-08 & 0.053461            & -0.152859        & 0.062083           \\
    2012-08-09 & -0.079313           & 0.245413         & 0.041363           \\
    \bottomrule
\end{tabular}

\subsection{b}
Covariance matrix of percentage returns (to be divided by 100*100 to get values in the unit of 1):

\begin{tabular}{lrrr}
    \toprule
              & Dow Jones & NASDAQ   & S\&P 500 \\
    \midrule
    Dow Jones & 0.617416  & 0.742765 & 0.711061 \\
    NASDAQ    & 0.742765  & 1.036670 & 0.879881 \\
    S\&P 500  & 0.711061  & 0.879881 & 0.826366 \\
    \bottomrule
\end{tabular}

\subsection{c}
Daily log returns:

\begin{tabular}{lrrr}
    \toprule
    Date       & Dow Jones Log Return & NASDAQ Log Return & S\&P 500 Log Return \\
    \midrule
    2012-07-26 & NaN                  & NaN               & NaN                 \\
    2012-07-27 & 0.014461             & 0.022163          & 0.018901            \\
    2012-07-30 & -0.0203              & -0.04150          & -0.0484             \\
    2012-07-31 & -0.04933             & -0.02148          & -0.04326            \\
    2012-08-01 & -0.02505             & -0.06591          & -0.02904            \\
    2012-08-02 & -0.07523             & -0.03581          & -0.07532            \\
    2012-08-03 & 0.016731             & 0.019781          & 0.018861            \\
    2012-08-06 & 0.01628              & 0.07389           & 0.02327             \\
    2012-08-07 & 0.03887              & 0.08642           & 0.05094             \\
    2012-08-08 & 0.0534               & -0.01530          & 0.0621              \\
    2012-08-09 & -0.0793              & 0.02451           & 0.0414              \\
    \bottomrule
\end{tabular}

\subsection{d}
Covariance matrix of log returns:

\begin{tabular}{lrrr}
    \toprule
              & Dow Jones & NASDAQ   & S\&P 500 \\
    \midrule
    Dow Jones & 0.000061  & 0.000073 & 0.000070 \\
    NASDAQ    & 0.000073  & 0.000102 & 0.000087 \\
    S\&P 500  & 0.000070  & 0.000087 & 0.000081 \\
    \bottomrule
\end{tabular}

\section{exercise 3}

\begin{tabular}{lrrrrrr}
    \toprule
          & $\omega_1$ & $\omega_2$ & $\omega_3$ & $\omega_4$ & $\omega_5$ & $\omega_6$ \\
    \midrule
    $P_0$ & 1.0305     & 1.0305     & 1.0305     & 1.0305     & 1.0305     & 1.0305     \\
    $P_1$ & 30.0       & 35.0       & 40.0       & 42.0       & 45.0       & 50.0       \\
    $P_2$ & 0.0        & 0.0        & 0.0        & 2.0        & 5.0        & 10.0       \\
    $P_3$ & 10.0       & 5.0        & 0.0        & 0.0        & 0.0        & 0.0        \\
    \bottomrule
\end{tabular}

Notation:

\begin{tabular}{ll}
    \toprule
          & Security                                \\
    \midrule
    $P_0$ & Cash (riskless), pays $e^{0.03}$ at $T$ \\
    $P_1$ & Underlying asset $S_T$                  \\
    $P_2$ & Six-month ATM call, $K=40 $             \\
    $P_3$ & Six-month ATM put, $K=40 $              \\
    \bottomrule
\end{tabular}

(ii) There are 6 states and 4 securities. There are more states than securities so it cannot be complete.

(iii) Note that due to the put-call parity, $P_1 - P_2 + P_3$ is $40/e^{0.03}$ times the first row, so one of the rows (securities) is redundant.
\end{document}
