\documentclass{article}
\usepackage{amsmath}
\usepackage{amssymb}
\usepackage{array}
\usepackage{graphicx} % Required for \scalebox
\usepackage{hyperref}
\usepackage{listings}
\usepackage{booktabs}
\usepackage{xcolor}
\usepackage{graphicx}
\usepackage{verbatim}
\usepackage[extreme]{savetrees} % tighter margins
\usepackage{lmodern} % Use scalable Latin Modern fonts
\title{Baruch SIP NLA Quiz 3}
\author{Daniel Tuzes}

\begin{document}
\maketitle

\section{exercise 1}
\[\begin{gathered}
        Av = \lambda v \hfill \\
        {A^7}v = {A^6}Av = {A^6}\lambda v =  \ldots  = {\lambda ^7}v \hfill \\
        {A^7}v = {A^8}v = A{A^7}v \hfill \\
        \left( {{\lambda ^7}v = {\lambda ^8}v,\,v \ne 0} \right) \Leftrightarrow {\lambda ^7} = {\lambda ^8} \Leftrightarrow \lambda  = 0{\text{ or }}\lambda  = 1 \hfill \\
    \end{gathered} \]

\section{exercise 2}
\[A\underbrace {Bv}_u = BAv = B\lambda v \Rightarrow Au = \lambda u\]
$\lambda$ is an eigenvalue with multiplicity 1, so
\[\begin{gathered}
        A\underbrace {Bv}_u = BAv = B\lambda v \Rightarrow Au = \lambda u \hfill \\
        u = cv\quad c \in \mathbb{R} \hfill \\
        Bv = u = cv \Rightarrow v{\text{ is eigenvector of }}B \hfill \\
    \end{gathered} \]
\section{exercise 3}
\[\left( {\begin{array}{*{20}{c}}
            1      & { - 1} \\
            { - 1} & 1
        \end{array}} \right)\left( {\begin{array}{*{20}{c}}
            1 \\
            1
        \end{array}} \right) = 0\left( {\begin{array}{*{20}{c}}
            1 \\
            1
        \end{array}} \right)\]
\[\left( {\begin{array}{*{20}{c}}
            1      & { - 1} \\
            { - 1} & 1
        \end{array}} \right)\left( {\begin{array}{*{20}{c}}
            1 \\
            { - 1}
        \end{array}} \right) = 2\left( {\begin{array}{*{20}{c}}
            1 \\
            -1
        \end{array}} \right)\]
So the 2 eigenvalues are 0 and 2, so it is a positive semidefinite matrix. And it is also symmetric.

\section{exercise 4}
Are all the leading principal minors positive?
\begin{enumerate}
    \item 1 is positive,
    \item $2.5-0.2\cdot0.2>0$ yes,
    \item it is $4.572>0$, see below
\end{enumerate}

\begin{verbatim}
import numpy as np

A = np.array([
    [1, -0.2, 0.8],
    [-0.2, 2.5, -1.3],
    [0.8, -1.3, 3.1]
])

print(np.linalg.det(A))

\end{verbatim}
\end{document}
