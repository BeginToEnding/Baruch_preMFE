\documentclass{article}
\usepackage{amsmath}
\usepackage{amssymb}
\usepackage{array}
\usepackage{graphicx} % Required for \scalebox
\usepackage{hyperref}
\usepackage{listings}
\usepackage{booktabs}
\usepackage{xcolor}
\usepackage{graphicx}
\usepackage{verbatim}
\usepackage[extreme]{savetrees} % tighter margins
\usepackage{lmodern} % Use scalable Latin Modern fonts
\title{Baruch SIP NLA Quiz 2}
\author{Daniel Tuzes}

\begin{document}
\maketitle

\section{exercise 1}
Multiply $L_1 U_1 = L_2 U_2$ by $U_2^{-1}$ from the right:

$$L_2 = L_1 U_1 U_2^{-1} := L_1 U_3$$

\begin{itemize}
    \item $U_3$ is upper triangluar: it is a product of 2 upper triangualar matrices: $U_1$ is upper triangular, and $U_2^{-1}$ as well: the inverse of an upper triangular is also upper triangular.
    \item $U_3$ must be lower triangular too, because LHS is lower triangular, and the RHS is the product of a lower triangular and $U_3$.
    \item Only the diagonal matrices are lower and upper triangular.
    \item Taking the determinant, on the LHS, it is non0, so it must be non0 on RHS too. $L_1$ is nonsingular, so $U_3$ must be nonsingular too.
\end{itemize}
$U_1 U_2^{-1} = U_3 = D^{-1}$, and the invese of a diagonal is also diagonal.

For the 2nd statement: $D U_1 = {D^{-1}}^{-1} U_1 = U_2 U_1^{-1} U_1 = U_2$.

\section{exercise 2}
\subsection{i/1}
Knowing that the inverse of a lower triangular is lower triangular, and product of lower triangulars is lower triangular:
\[LA = L' \Rightarrow {L^{ - 1}}LA = {L^{ - 1}}L' \Rightarrow A = L''\]

If we don't know:
\[\left( {{{\left( {LA} \right)}_{ik}} = {L_{ij}}{A_{jk}} = 0{\text{ if }}i < k} \right)\mathop  \Rightarrow \limits^? \left( {{A_{jk}} = 0{\text{ if }}j < k} \right)\]
We use mathematical induction. For $i=1$,
\[\left( {{{\left( {LA} \right)}_{1k}} = {L_{1j}}{A_{jk}}\mathop  = \limits^{L{\text{ is lower tri}}{\text{.}}} {L_{1,1}}{A_{1k}} = 0{\text{ if }}1 < k} \right) \Rightarrow \left( {A_{1k}} = 0{\text{ if }}k > 1\right)\]
Suppose that $A_{jk}=0 \text{ if }j<k$ is satisfied till some $j',\quad j\le j'$, does it imply than that it is true for $j+1$ too?

Consider a $k>j'+1$, so $A_{jk} = 0$ for $j < j'+1$:
\[\left( {{{\left( {LA} \right)}_{j' + 1,k}} = {L_{j' + 1,i}}{A_{ik}}\mathop  = \limits^{L{\text{ is l. tri}}{\text{.}}} \sum\limits_{i = 1}^{j' + 1} {{L_{j' + 1,i}}{A_{ik}}} \mathop  = \limits^{A{\text{ is l. tri}}{\text{. till row }}j'} {L_{j' + 1,j' + 1}}{A_{j' + 1,k}} = 0{\text{ if }}j' + 1 < k} \right) \Rightarrow \left( {A_{j' + 1,k}} = 0{\text{ if }}k > j' + 1\right)\]
So $A_{jk}$ is triangluar in row $j'+1$ too.
\subsection{i/2}
\[AL = L' \Rightarrow AL{L^{ - 1}} = L'{L^{ - 1}} \Rightarrow A = L''\]

If we don't know, we can use induction again,
starting with the last column of $A$ based on the fact that the last column of the product is 0 except in the last row.
Then if $A$ is triangular between column $n$ and $j'$, it can be shown that is triangluar in row $j'-1$ too.


\subsection{ii}
Transpose the i/1 and i/2 to get ii/2 and ii/1, resepctively.

\section{exercise 3}
Python is pretty standard and simple, better than inventing my own pseudo code syntax
\begin{verbatim}
def forward_subst_Lband2(L, b, n):
    """Return the solution for Ly = b if L is a lower triangular matrix with band 2."""
    
    y = [0] * n  # assignments are free
    # first element
    y[0] = b[0] / L[0][0]  # 1

    # the rest are recursive
    # condition checks and iterators are free
    for i in range(1, n): # (n-1) * (4 + 2) ~ 6n
        
        # we scalar product the row of lower banded matrix,
        # which are non0 only till the diagonal element, and the 2 elementes before,
        # with the known elements from y
        partial_sum = 0
        for j in range(max(0, i - 2), i):  # 2x2 if j>1, 1 if j=1 ~ 4
            partial_sum += L[i][j] * y[j]  # 2
        
        y[i] = ((b[i] - partial_sum) / L[i][i])  # 2
    return y
\end{verbatim}

The opcount is $6n + \mathcal{O}(1)$.

We can inline the 2 additions for partial sum,
eliminating 1 addtion,
leading to an opcount to $5n + \mathcal O (1)$. Adding anything to 0, however,
is not only cheaper than regular addition,
but must probably a C++ compiler would optimize this part out.

\section{exercise 4}

\begin{verbatim}
def backw_subst_Uband2(U, b, n):
    """Return the solution for Uy = b if U is an upper triangular matrix with band 2."""

    y = [0] * n  # assignments are free
    # last element
    y[n-1] = b[n-1] / U[n-1][n-1]  # 1  # index argument calculations are free

    # the rest are recursive
    # condition checks and iterators are free
    for i in range(n-2,-1, -1):  # (n-1) * (4 + 2) ~ 6n

        # we scalar product the row of upper banded matrix,
        # which are non0 only from the diagonal till 2 more elements,
        # with the known elements from y
        partial_sum = 0
        for j in range(i+1,min(n,i+3)):  # 2x2 if j<n-2, 1 if j=n-2 ~ 4
            partial_sum += U[i][j] * y[j]  # 2

        y[i] = ((b[i] - partial_sum) / U[i][i])  # 2

    return y
\end{verbatim}

\end{document}
