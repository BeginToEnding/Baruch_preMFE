\documentclass{article}
\usepackage{amsmath}
\usepackage{amssymb}
\usepackage{array}
\usepackage{graphicx} % Required for \scalebox
\usepackage{hyperref}
\usepackage{listings}
\usepackage{xcolor}
\usepackage{graphicx}
\usepackage{booktabs}
\usepackage{verbatim}
\usepackage[extreme]{savetrees} % tighter margins
\title{Baruch ACDE HW 3}
\author{group 1}

\begin{document}
\maketitle
\tableofcontents

\section{exercise 1 Daniel}
Make substitution of \[\ln \left( x \right)=-t\] so
\[\lim ...=\underset{t\to \infty }{\mathop{\lim }}\,{{e}^{\overbrace{t{{c}_{3}}-{{\left( {{c}_{1}}t-{{c}_{2}} \right)}^{2}}}^{\to -\infty }}}=0\]

\section{exercise 2}

\section{exercise 3}

\section{exercise 4}

\section{exercise 5}
\section{exercise 6}
\section{exercise 7}
\section{exercise 8}
\section{exercise 9}
\section{exercise 10}
\section{exercise 11}
\section{exercise 12}
\begin{tabular}{lrrr}
    \toprule
        & n\_0.1          & n\_0.5          & n\_1.0          \\
    \midrule
    4   & 0.5398278375347 & 0.6914631235012 & 0.8413554878566 \\
    8   & 0.5398278372931 & 0.6914625023982 & 0.8413454061391 \\
    16  & 0.5398278372780 & 0.6914624638402 & 0.8413447871501 \\
    32  & 0.5398278372771 & 0.6914624614343 & 0.8413447486335 \\
    64  & NaN             & 0.6914624612840 & 0.8413447462288 \\
    128 & NaN             & 0.6914624612746 & 0.8413447460786 \\
    256 & NaN             & 0.6914624612741 & 0.8413447460692 \\
    512 & NaN             & NaN             & 0.8413447460686 \\
    \bottomrule
\end{tabular}

Nan means calculation was not required as convergence was reached.

\section{exercise 13}

\begin{tabular}{rrrrr}
    \toprule
    n  & D(0.5)     & D(1.0)     & D(1.5)     & D(2.0)     \\
    \midrule
    2  & 0.98256846 & 0.96043413 & 0.93671482 & 0.91340548 \\
    4  & 0.98256855 & 0.96051378 & 0.93713849 & 0.91398132 \\
    8  & NaN        & 0.96051612 & 0.93714794 & 0.91403997 \\
    16 & NaN        & 0.96051626 & 0.93714845 & 0.91404123 \\
    32 & NaN        & NaN        & NaN        & 0.91404131 \\
    \bottomrule
\end{tabular}

Nan means calculation was not required as convergence was reached.
\section{exercise 14}

\textbf{Objective:} Compute the value of a European-style option with the payoff
\[
    \sqrt{\ln\left(\frac{S(T)}{K}\right)} \cdot \mathbf{1}_{\{S(T) > K\}}
\]
at maturity \( T = 0.25 \).

\vspace{1em}

\textbf{Step 1: Risk-neutral valuation formula}
\[
    V = e^{-rT} \mathbb{E}^{\mathbb{Q}}\left[\sqrt{\ln\left(\frac{S(T)}{K}\right)} \cdot \mathbf{1}_{\{S(T) > K\}} \right]
\]

\vspace{1em}

\textbf{Step 2: Distribution of \( S(T) \) under risk-neutral measure}

Assuming the asset pays a continuous dividend yield \( q \), volatility \( \sigma \), and follows a geometric Brownian motion under the risk-neutral measure:
\[
    S(T) = S(0) \exp\left((r - q - \tfrac{1}{2}\sigma^2)T + \sigma \sqrt{T} Z\right), \quad Z \sim \mathcal{N}(0, 1)
\]

Taking logarithms:
\[
    \ln\left(\frac{S(T)}{K}\right) = \ln\left(\frac{S(0)}{K}\right) + (r - q - \tfrac{1}{2}\sigma^2)T + \sigma \sqrt{T} Z
\]

Define:
\[
    X := \ln\left(\frac{S(T)}{K}\right) \sim \mathcal{N}(\mu, \sigma_X^2)
\]
where
\[
    \mu = \ln\left(\frac{S(0)}{K}\right) + (r - q - \tfrac{1}{2}\sigma^2)T,\quad \sigma_X^2 = \sigma^2 T
\]

Then the value becomes:
\[
    V = e^{-rT} \mathbb{E}^{\mathbb{Q}}\left[\sqrt{X} \cdot \mathbf{1}_{\{X > 0\}} \right]
    = e^{-rT} \int_0^{\infty} \sqrt{x} \cdot f_X(x) \, dx
\]
where
\[
    f_X(x) = \frac{1}{\sqrt{2\pi \sigma_X^2}} \exp\left( -\frac{(x - \mu)^2}{2 \sigma_X^2} \right)
\]

\vspace{1em}

\textbf{Step 3: Suggested substitution}

Use the substitution \( x \in [0, \infty) \rightarrow y \in (0, 1] \) via
\[
    y = e^{-x} \quad \Rightarrow \quad x = -\ln y, \quad dx = -\frac{1}{y} dy
\]

Then the integral becomes:
\[
    V = e^{-rT} \int_0^1 \sqrt{-\ln y} \cdot \frac{1}{y \sqrt{2\pi \sigma_X^2}}
    \exp\left( -\frac{(-\ln y - \mu)^2}{2\sigma_X^2} \right) dy
\]

\begin{table}[h!]
    \centering
    \begin{tabular}{ccc}
        \toprule
        \textbf{n} & \textbf{Value} & \textbf{Difference} \\
        \midrule
        4          & 0.252179245276 & {}                  \\
        8          & 0.284557236478 & 3.26e-02            \\
        16         & 0.281420173997 & 3.16e-03            \\
        32         & 0.283062151035 & 1.65e-03            \\
        64         & 0.283681879287 & 6.24e-04            \\
        128        & 0.283907525376 & 2.27e-04            \\
        256        & 0.283988416587 & 8.15e-05            \\
        512        & 0.284017211120 & 2.90e-05            \\
        1024       & 0.284027425999 & 1.03e-05            \\
        2048       & 0.284031043601 & 3.64e-06            \\
        4096       & 0.284032323695 & 1.29e-06            \\
        8192       & 0.284032776467 & 4.56e-07            \\
        16384      & 0.284032936580 & 1.61e-07            \\
        32768      & 0.284032993194 & 5.70e-08            \\
        65536      & 0.284033013212 & 2.02e-08            \\
        131072     & 0.284033020289 & 7.13e-09            \\
        262144     & 0.284033022791 & 2.52e-09            \\
        524288     & 0.284033023676 & 8.91e-10            \\
        \bottomrule
    \end{tabular}
    \caption{Convergence of option value using Simpson's Rule for increasing \( n \)}
\end{table}

\section{exercise 15}

\end{document}
