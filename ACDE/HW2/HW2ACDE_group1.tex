\documentclass{article}
\usepackage{amsmath}
\usepackage{amssymb}
\usepackage{array}
\usepackage{graphicx} % Required for \scalebox
\usepackage{hyperref}
\usepackage{listings}
\usepackage{xcolor}
\usepackage{graphicx}
\usepackage{booktabs}
\usepackage{verbatim}
\usepackage[extreme]{savetrees} % tighter margins
\title{Baruch ACDE HW 2}
\author{group 1}

\begin{document}
\maketitle
\tableofcontents
\section{exercise 1}
\subsection{i Hao}
\subsection{ii Hongjun}
\subsection{iii Daniel}
The value of a put option is:
\[P(0) = K e^{-rT} N(-d_2) - S(0) N(-d_1)\]
where
\begin{align*}
    d_1 & = \frac{\ln\left(\frac{S(0)}{K}\right) + \left(r + \frac{\sigma^2}{2}\right)T}{\sigma\sqrt{T}} \\
    d_2 & = d_1 - \sigma\sqrt{T}
\end{align*}
Calculating the value of the put option with spots 100 and 102,
strike 100, $\sigma = 0.30$, $r = 0.05$, and $T = 0.5$,
the difference for 1000 options is $786.66USD$.

Considering that 1 day has passed, and a year has 252 trading days,
we can calculate the new value with $T = 0.5 - \frac{1}{252}$.
The difference for 1000 options is $787.20USD$.

Sanity check: the $\Delta$ of a put option tends to $-0.5$ as $T$ approaches 0.
With $T=0.5$ at the beginning, the $\Delta$ is around $-0.4114$.
The negative sign shows that the value of the put option decreases as the stock price increases,
and indeed, we lost some value as the stock price increased from 100 to 102.
And after a day, when $\Delta$ is a bit closer to $-0.5$,
the value of the put option decreased a bit more.


\section{exercise 2}
\subsection{i Daniel}


\begin{align*}
    d_1                    & = \frac{\ln(S/K) + \left(r + \frac{1}{2} \sigma^2\right) T}{\sigma \sqrt{T}} \\
    d_2                    & = d_1 - \sigma \sqrt{T}                                                      \\
    P                      & = K e^{-rT} N(-d_2) - S N(-d_1)                                              \\
    \text{Portfolio Value} & = 2000 \cdot P + 400 \cdot S + 10,\!000
\end{align*}

After calculating the value of the portfolio with the given parameters:
\begin{itemize}
    \item Put price: $5.4076USD$
    \item Value of puts: $10815.14USD$
    \item Value of stock: $14000.00USD$
\end{itemize}

So the portfolio value is $34815.14USD$.

\subsection{ii Daniel}
We can calculate the $\Delta$ of the portfolio, and
\begin{itemize}
    \item sell or buy the asset to make it delta-neutral
    \item sell or buy puts
    \item sell of buy calls
\end{itemize}
Shorting the asset and puts are also possibilities, but as we have a long position in them, to save on the expenses,
we will not consider them. Introducing calls would add complexity to the portfolio,
so we will not consider them either. In the standard delta-hedging strategies,
we operate with the amount of the underlying assets, so we'll buy or sell the asset to make the portfolio delta-neutral.

Strategy: calculate the $\Delta$ of the portfolio, and buy or sell the asset to make it delta-neutral.
The $\Delta$ of the asset is 1, and the $\Delta$ of the put is $$N(d_1)-1 = -0.78285$$
So we need to own 2000 times the $\Delta$ in shares, i.e. we need to buy further stock in the amount of $-1165.71$.
Let's suppose that fractional shares are allowed, so we can buy $1165.71$ shares.
To do that, we need $1165.71 \cdot 35 = 40900USD$, out which $10,000$ is already in the portfolio.
So we need to borrow $30900USD$ to buy the shares.
\subsection{iii Daniel}

\begin{tabular}{lrrrrrrrr}
    \toprule
              & puts & assets & cash        & $S$ & $T$    & $\Delta_{1P}$ & $\Delta_{puts}$ & $\Delta_{total}$ \\
    \midrule
    week 0 BH & 2000 & 400    & 10000.0000  & 35  & 0.2500 & -0.7829       & -1565.7125      & -1165.7125       \\
    week 0 AH & 2000 & 1566   & -30810.0000 & 35  & 0.2500 & -0.7829       & -1565.7125      & 0.2875           \\
    week 1 BH & 2000 & 1566   & -30821.8523 & 40  & 0.2308 & -0.4586       & -917.1032       & 648.8968         \\
    week 1 AH & 2000 & 917    & -4861.8523  & 40  & 0.2308 & -0.4586       & -917.1032       & -0.1032          \\
    week 2 BH & 2000 & 917    & -4863.7226  & 36  & 0.2115 & -0.7466       & -1493.2718      & -576.2718        \\
    week 2 AH & 2000 & 1493   & -25599.7226 & 36  & 0.2115 & -0.7466       & -1493.2718      & -0.2718          \\
    week 3 BH & 2000 & 1493   & -25609.5705 & 32  & 0.1923 & -0.9453       & -1890.6535      & -397.6535        \\
    week 3 AH & 2000 & 1891   & -38345.5705 & 32  & 0.1923 & -0.9453       & -1890.6535      & 0.3465           \\
    week 4 BH & 2000 & 1891   & -38360.3217 & 37  & 0.1731 & -0.7035       & -1407.0149      & 483.9851         \\
    week 4 AH & 2000 & 1407   & -20452.3217 & 37  & 0.1731 & -0.7035       & -1407.0149      & -0.0149          \\
    \bottomrule
\end{tabular}

\begin{tabular}{|c|c|}
    \hline
    \textbf{Week} & \textbf{Asset Change at AH} \\
    \hline
    Week 0        & +1166                       \\
    Week 1        & -649                        \\
    Week 2        & +974                        \\
    Week 3        & -983                        \\
    Week 4        & +479                        \\
    \hline
\end{tabular}


\section{exercise 3 Hao}
\subsection{i}
Compute deltas for call and put options:
\[
\Delta_C = e^{-qT}N(d_1), \quad \Delta_P = -e^{-qT}N(-d_1)
\]
where
\[
d_1 = \frac{\ln(S_0/K)+(r-q+\sigma^2/2)T}{\sigma\sqrt{T}}
\]

For call (K=55, T=0.5):
\[
d_1 = \frac{\ln(50/55)+0.03}{0.2\sqrt{0.5}} \approx -0.741 \Rightarrow \Delta_C \approx 0.227
\]

For put (K=45, T=0.5):
\[
d_1 = \frac{\ln(50/45)+0.03}{0.2\sqrt{0.5}} \approx 0.911 \Rightarrow \Delta_P \approx -0.179
\]

Portfolio delta:
\[
1000\times0.227 + 600\times(-0.179) \approx 119.6
\]

Hedge: Short 119.6 shares.

Cash borrowed:
\[
\text{Cash} = 119.6 \times 50 = 5980
\]

\subsection{ii}

Gamma for all options:
\[
\Gamma = \frac{e^{-qT}N'(d_1)}{S_0\sigma\sqrt{T}} \approx 0.042
\]

Portfolio gamma:
\[
1000\times0.042 + 600\times0.042 = 67.2
\]

Compute delta for ATM call (K=50, T=0.75):
\[
d_1^{ATM} = \frac{\ln(50/50)+(0.04-0.02+0.2^2/2)0.75}{0.2\sqrt{0.75}} \approx 0.306
\]
\[
\Delta_{ATM} = e^{-qT}N(d_1) = e^{-0.015}\times0.620 \approx 0.62
\]

Gamma for ATM call:
\[
\Gamma_{ATM} = \frac{e^{-0.015}\times\frac{1}{\sqrt{2\pi}}e^{-0.306^2/2}}{50\times0.2\sqrt{0.75}} \approx 0.035
\]

Gamma hedge using 9M ATM call ($\Gamma_{ATM}\approx0.035$):
\[
\text{Short } \frac{67.2}{0.035} \approx 1920 \text{ calls}
\]

New delta after gamma hedge:
\[
119.6 - 1920\times0.62 \approx -1070.8
\]

Final hedge: Long 1070.8 shares.

\subsection{iii}
\subsubsection*{Original Option Prices (S=50, $\sigma$=0.2)}

Call (K=55):
\[
d_1 = \frac{\ln(50/55)+(0.04-0.02+0.2^2/2)0.5}{0.2\sqrt{0.5}} \approx -0.741
\]
\[
N(d_1) \approx 0.229,\ N(d_2) \approx 0.189
\]
\[
C = 50e^{-0.01}\times0.229 - 55e^{-0.02}\times0.189 \approx 1.21
\]

Put (K=45):
\[
d_1 = \frac{\ln(50/45)+0.03}{0.2\sqrt{0.5}} \approx 0.911
\]
\[
N(-d_1) \approx 0.181,\ N(-d_2) \approx 0.221
\]
\[
P = 45e^{-0.02}\times0.221 - 50e^{-0.01}\times0.181 \approx 0.89
\]

\subsubsection*{New Option Prices (S=54, $\sigma$=0.3)}

Call (K=55):
\[
d_1 = \frac{\ln(54/55)+(0.04-0.02+0.3^2/2)0.5}{0.3\sqrt{0.5}} \approx -0.078
\]
\[
N(d_1) \approx 0.469,\ N(d_2) \approx 0.386
\]
\[
C_{new} = 54e^{-0.01}\times0.469 - 55e^{-0.02}\times0.386 \approx 2.86
\]

Put (K=45):
\[
d_1 = \frac{\ln(54/45)+0.065}{0.3\sqrt{0.5}} \approx 1.428
\]
\[
N(-d_1) \approx 0.077,\ N(-d_2) \approx 0.112
\]
\[
P_{new} = 45e^{-0.02}\times0.112 - 54e^{-0.01}\times0.077 \approx 0.38
\]

\subsubsection*{9M ATM Call Prices}

Original (S=50, $\sigma$=0.2):
\[
d_1 = \frac{0+0.03\times0.75}{0.2\sqrt{0.75}} \approx 0.306
\]
\[
C_{ATM} = 50e^{-0.015}\times0.620 - 50e^{-0.03}\times0.559 \approx 2.47
\]

New (S=54, $\sigma$=0.3):
\[
d_1 = \frac{\ln(54/50)+0.065\times0.75}{0.3\sqrt{0.75}} \approx 0.793
\]
\[
C_{ATM}^{new} = 54e^{-0.015}\times0.786 - 50e^{-0.03}\times0.714 \approx 6.38
\]

\subsubsection*{Portfolio Value Changes}

Initial portfolio value change:
\[
1000 \times (2.86 - 1.21) + 600 \times (0.38 - 0.89) = +1344
\]

Delta-hedged P\&L:
\[
1344 - 119.6 \times (54 - 50) = +865.6
\]

Delta+gamma-hedged P\&L:
\[
1344 - 1920\times(6.38-2.47) + 1070.8\times4 = -1880
\]


\section{exercise 4 Hao}

Let $N(x)$ be the CDF of standard normal distribution with $N(0)=0.5$.

\subsubsection*{Forward Difference Approximation}
The forward difference approximation:
\[
N'(0) \approx \frac{N(h)-N(0)}{h}
\]

Taylor expansion of $N(h)$:
\[
N(h) = N(0) + N'(0)h + \frac{N''(0)}{2}h^2 + \frac{N'''(0)}{6}h^3 + \mathcal{O}(h^4)
\]

Derivatives at $x=0$:
\[
N'(0) = \frac{1}{\sqrt{2\pi}}, \quad N''(0)=0, \quad N'''(0)=-\frac{1}{\sqrt{2\pi}}
\]

Substituting:
\[
\frac{N(h)-N(0)}{h} = \frac{1}{\sqrt{2\pi}} - \frac{h^2}{6\sqrt{2\pi}} + \mathcal{O}(h^4)
\]
Error term:
\[
\left|\frac{N(h)-N(0)}{h}-N'(0)\right| = \mathcal{O}(h^2)
\]

\subsubsection*{Central Difference Approximation}
The central difference approximation:
\[
N'(0) \approx \frac{N(h)-N(-h)}{2h}
\]

Taylor expansions:
\[
N(h) = 0.5 + \frac{h}{\sqrt{2\pi}} - \frac{h^3}{6\sqrt{2\pi}} + \mathcal{O}(h^5)
\]
\[
N(-h) = 0.5 - \frac{h}{\sqrt{2\pi}} + \frac{h^3}{6\sqrt{2\pi}} + \mathcal{O}(h^5)
\]

Substituting:
\[
\frac{N(h)-N(-h)}{2h} = \frac{1}{\sqrt{2\pi}} - \frac{h^2}{6\sqrt{2\pi}} + \mathcal{O}(h^4)
\]
Error term:
\[
\left|\frac{N(h)-N(-h)}{2h}-N'(0)\right| = \mathcal{O}(h^2)
\]

\subsubsection*{Conclusions}
\begin{itemize}
\item Forward difference is $\mathcal{O}(h^2)$ at $x=0$ (normally $\mathcal{O}(h)$)
\item Central difference remains $\mathcal{O}(h^2)$
\item Both have same leading error term $-\frac{h^2}{6\sqrt{2\pi}}$
\end{itemize}


\section{exercise 5}
\subsection{i Daniel}
\begin{align}
     & f\left( a+h \right) = f\left( a \right)+f'\left( a \right)h+\frac{f''\left( a \right){{h}^{2}}}{2}+O\left( {{h}^{3}} \right)\quad \text{just reminder}                                                                           \\
     & {{f}_{2}}:=f\left( a+2h \right) = f\left( a \right)+f'\left( a \right)2h+\frac{f''\left( a \right)4{{h}^{2}}}{2}+O\left( {{h}^{3}} \right):={{f}_{0}}+f{{'}_{0}}\cdot 2h+f'{{'}_{0}}\cdot 2{{h}^{2}}+O\left( {{h}^{3}} \right)   \\
     & {{f}_{3}}:=f\left( a+3h \right) = f\left( a \right)+f'\left( a \right)3h+\frac{f''\left( a \right)9{{h}^{2}}}{2}+O\left( {{h}^{3}} \right):={{f}_{0}}+f{{'}_{0}}\cdot 3h+f'{{'}_{0}}\cdot 4.5{{h}^{2}}+O\left( {{h}^{3}} \right) \\
     & 9{{f}_{2}}-4{{f}_{3}}=5{{f}_{0}}+6f{{'}_{0}}h+O\left( {{h}^{3}} \right)                                                                                                                                                          \\
\end{align}
Therefore:
\[f'\left( a \right)=f{{'}_{0}}=\frac{9{{f}_{2}}-4{{f}_{3}}-5{{f}_{0}}}{6h}+O\left( {{h}^{2}} \right)=\frac{9f\left( a+2h \right)-4f\left( a+3h \right)-5f\left( a \right)}{6h}+O\left( {{h}^{2}} \right)\]
\subsection{ii Daniel}

\begin{align}
     & {{f}_{1}}:=f\left( a+h \right)={{f}_{0}}+f{{'}_{0}}h+\frac{f'{{'}_{0}}{{h}^{2}}}{2}+f''{{'}_{0}}\frac{{{h}^{3}}}{6}+O\left( {{h}^{4}} \right)\quad \text{just reminder} \\
     & {{f}_{-1}}:=f\left( a-h \right)={{f}_{0}}-f{{'}_{0}}\cdot h+\frac{f'{{'}_{0}}\cdot {{h}^{2}}}{2}-f''{{'}_{0}}\frac{{{h}^{3}}}{6}+O\left( {{h}^{4}} \right)              \\
     & {{f}_{2}}:=f\left( a+2h \right)={{f}_{0}}+f{{'}_{0}}\cdot 2h+f'{{'}_{0}}\cdot 2{{h}^{2}}+f''{{'}_{0}}\frac{8{{h}^{3}}}{6}+O\left( {{h}^{4}} \right)                     \\
     & {{f}_{4}}:=f\left( a+4h \right)={{f}_{0}}+f{{'}_{0}}\cdot 4h+f'{{'}_{0}}\cdot 8{{h}^{2}}+f''{{'}_{0}}\frac{64{{h}^{3}}}{6}+O\left( {{h}^{4}} \right)                    \\
\end{align}

Take $A$ of $f_{-1}$ and $B$ of $f_{2}$ and $C$ of $f_4$,
then we want to eliminate $f'_0$ and $f'''_0$.

With a choice of $A=-16$, $B=-10$ and $C=1$, and summing them uo, eliminate the unwanted terms:
\begin{itemize}
    \item $f'_0 [-(-16) + (-10)2 + 1\cdot 4] = 0$.
    \item $f'''_0 [(-16)\cdot \frac{-1}{6} + (-10)\cdot \frac{8}{6} + 1\cdot \frac{64}{6}] = 0$.
\end{itemize}
What remained:
\begin{itemize}
    \item $f_0 [(-16) + (-10) + 1] = -25f_0$.
    \item $f''_0 [(-16)\cdot \frac{1}{2} + (-10)\cdot 2 + 1\cdot 8] = -20f''_0$.
\end{itemize}

So the result is
\[f''\left( a \right)=\frac{16f\left( a-h \right)+10f\left( a+2h \right)-f\left( a+4h \right)-25f\left( a \right)}{20{{h}^{2}}}+O\left( {{h}^{2}} \right)\]
\subsection{iii Hao}

\section{exercise 6 Hongjun}

\section{exercise 7 Daniel}
The payoff defined in the exercise is not in the units of currency. In the solution I suppose that the payoff is

\[{{\Pi }_{7}}\left( T \right)=\max \left( \frac{S{{\left( T \right)}^{2}}}{{{\$}^{2}}}-\frac{K}{\$},0 \right)\quad S\left( T \right)=S\left( 0 \right){{e}^{\left( r-q-\frac{{{\sigma }^{2}}}{2} \right)T+\sigma \sqrt{T}\cdot Z}}\]
where $\$$ denotes the currency and omitted in the exercise later. The value of the option at time $T$ is given by the risk-neutral expectation of the payoff:

\[\begin{matrix}
        {{V}_{7}}\left( T \right)={{E}_{RN}}\left( {{\Pi }_{7}}\left( T \right) \right)=\int_{S{{\left( T \right)}^{2}}>K}{\left( S{{\left( T \right)}^{2}}-K \right)d\mathsf{\mathcal{P}}}\quad S{{\left( T \right)}^{2}}>K\Leftrightarrow & S{{\left( 0 \right)}^{2}}{{e}^{2\left[ \left( r-q-\frac{{{\sigma }^{2}}}{2} \right)T+\sigma \sqrt{T}\cdot Z \right]}}>K                               \\
        {}                                                                                                                                                                                                                                  & \ln \left( \frac{K}{S{{\left( 0 \right)}^{2}}} \right)<2\left[ \left( r-q-\frac{{{\sigma }^{2}}}{2} \right)T+\sigma \sqrt{T}\cdot Z \right]           \\
        {}                                                                                                                                                                                                                                  & Z>\frac{\frac{1}{2}\ln \left( \frac{K}{S{{\left( 0 \right)}^{2}}} \right)-\left( r-q-\frac{{{\sigma }^{2}}}{2} \right)T}{\sigma \sqrt{T}}:=-{{d}_{2}} \\
    \end{matrix}\]
Here we model the stock price as a geometric Brownian motion, and $Z$ is a standard normal random variable. The risk-neutral expectation is given by the integral over the probability density function of $Z$.
\[{{V}_{7}}\left( T \right)=\int\limits_{-{{d}_{2}}}^{\infty }{{{e}^{-\frac{1}{2}{{z}^{2}}}}\left( S{{\left( T \right)}^{2}}-K \right)dz}\]
\[{{V}_{7}}\left( T \right)=-K\cdot N\left( {{d}_{2}} \right)+\int\limits_{-{{d}_{2}}}^{\infty }{{{e}^{-\frac{1}{2}{{z}^{2}}}}S{{\left( T \right)}^{2}}\frac{dz}{\sqrt{2\pi }}}\]

Now let's take care about the integral with $S$ inside.
\begin{align}
     & I=\int\limits_{-{{d}_{2}}}^{\infty }{{{e}^{-\frac{1}{2}{{z}^{2}}}}S{{\left( T \right)}^{2}}\frac{dz}{\sqrt{2\pi }}}=\int\limits_{-{{d}_{2}}}^{\infty }{{{e}^{-\frac{1}{2}{{z}^{2}}}}S{{\left( 0 \right)}^{2}}{{e}^{2\left[ \left( r-q-\frac{{{\sigma }^{2}}}{2} \right)T+\sigma \sqrt{T}\cdot z \right]}}\frac{dz}{\sqrt{2\pi }}} \\
     & I=S{{\left( 0 \right)}^{2}}{{e}^{2\left( r-q-\frac{{{\sigma }^{2}}}{2} \right)T}}\int\limits_{-{{d}_{2}}}^{\infty }{{{e}^{-\frac{1}{2}{{z}^{2}}+2\sigma \sqrt{T}\cdot z}}\frac{dz}{\sqrt{2\pi }}}\qquad -\frac{1}{2}{{z}^{2}}+2\sigma \sqrt{T}\cdot z=-\frac{1}{2}{{\left( z-2\sigma \sqrt{T} \right)}^{2}}+2{{\sigma }^{2}}T     \\
     & I=S{{\left( 0 \right)}^{2}}{{e}^{2\left( r-q-\frac{{{\sigma }^{2}}}{2} \right)T+2{{\sigma }^{2}}T}}\int\limits_{-{{d}_{2}}}^{\infty }{{{e}^{-\frac{1}{2}{{\left( z-2\sigma \sqrt{T} \right)}^{2}}}}\frac{dz}{\sqrt{2\pi }}}\qquad z-2\sigma \sqrt{T}:=\hat{z}\quad -{{d}_{2}}-2\sigma \sqrt{T}:=-{{{\hat{d}}}_{1}}                \\
     & I=S{{\left( 0 \right)}^{2}}{{e}^{2\left( r-q+\frac{{{\sigma }^{2}}}{2} \right)T}}\int\limits_{-{{{\hat{d}}}_{1}}}^{\infty }{{{e}^{-\frac{1}{2}{{{\hat{z}}}^{2}}}}\frac{d\hat{z}}{\sqrt{2\pi }}}                                                                                                                                   \\
     & I=S{{\left( 0 \right)}^{2}}{{e}^{2\left( r-q+\frac{{{\sigma }^{2}}}{2} \right)T}}N\left( {{{\hat{d}}}_{1}} \right)                                                                                                                                                                                                                \\
\end{align}
We put it back to the equation for $V_7(T)$, and discount it to $t=0$.
\[{{V}_{7}}\left( 0 \right)=-K\cdot N\left( {{d}_{2}} \right) e^{-rT} +S{{\left( 0 \right)}^{2}}{{e}^{\left( r - 2q + {\sigma }^{2} \right)T}}N\left( {{{\hat{d}}}_{1}} \right)\]
Note that $\hat{d}_1$ is defined differently than in the standard Black-Scholes formula.


\section{exercise 8 Hao}
The value of a European cash-or-nothing put option is given by:
\[
P_{CoN} = B e^{-rT} N(-d_2),
\]
where:
\[
d_2 = \frac{\ln(S/K) + (r - q - \sigma^2/2)T}{\sigma \sqrt{T}}.
\]

The Delta ($\Delta$) of the option is the partial derivative of $P_{CoN}$ with respect to $S$:
\[
\Delta_{CoN} = \frac{\partial P_{CoN}}{\partial S} = B e^{-rT} \cdot \frac{\partial N(-d_2)}{\partial S}.
\]

Since $N(-d_2)$ is the CDF of the standard normal distribution, its derivative is the PDF:
\[
\frac{\partial N(-d_2)}{\partial S} = -n(d_2) \cdot \frac{\partial d_2}{\partial S},
\]
where $n(d_2) = \frac{1}{\sqrt{2\pi}} e^{-d_2^2/2}$.

The partial derivative of $d_2$ with respect to $S$ is:
\[
\frac{\partial d_2}{\partial S} = \frac{1}{S \sigma \sqrt{T}}.
\]

Substituting this into the Delta formula:
\[
\Delta_{CoN} = B e^{-rT} \cdot \left( -n(d_2) \cdot \frac{1}{S \sigma \sqrt{T}} \right) = -\frac{B e^{-rT} n(d_2)}{S \sigma \sqrt{T}}.
\]

The final Delta formula is:
\[
\Delta_{CoN} = -\frac{B e^{-rT}}{S \sigma \sqrt{2\piT}} \cdot e^{-d_2^2/2}.
\]


\section{exercise 9 Hongjun}

\end{document}
