\documentclass{article}
\usepackage{amsmath}
\usepackage{amssymb}
\usepackage{array}
\usepackage{graphicx} % Required for \scalebox
\usepackage{hyperref}
\usepackage{listings}
\usepackage{xcolor}
\usepackage{graphicx}
\usepackage{booktabs}
\usepackage{verbatim}
\usepackage[extreme]{savetrees} % tighter margins
\title{Baruch ACDE HW 4}
\author{group 1}

\begin{document}
\maketitle
\tableofcontents

\section{exercise 1 Daniel}
The derivative of the $\Delta$ is needed for the Newton method:

\[
    N(x) = \int_{-\infty}^x \frac{1}{\sqrt{2\pi}} e^{-t^2/2} \, dt
\]

\[
    \frac{d}{dK} \Delta_{\text{call}}(K)
    = \frac{d}{dK} \left( e^{-qT} N(d_1) \right)
    = e^{-qT} \cdot \frac{dN(d_1)}{dK}
    = e^{-qT} \cdot \frac{1}{\sqrt{2\pi}} e^{-d_1^2/2} \cdot \frac{d d_1}{dK}
\]

\[
    d_1 = \frac{\ln(S/K) + (r - q + \frac{1}{2} \sigma^2)T}{\sigma \sqrt{T}}, \quad
    \Rightarrow \frac{d d_1}{dK} = -\frac{1}{K \sigma \sqrt{T}}
\]

\[
    \Rightarrow \frac{d}{dK} \Delta_{\text{call}}(K)
    = -\frac{e^{-qT}}{K \sigma \sqrt{2\pi T}} e^{-d_1^2/2}
\]

\begin{table}
    \centering
    \caption{Newton Method Iterations for Finding Strike starting from ATM}
    \label{tab:newton_iterations}
    \begin{tabular}{rrrrr}
        \toprule
        K         & Delta(K) & dDelta/dK & Next K    & Error    \\
        \midrule
        30.000000 & 0.538480 & -0.087991 & 30.437315 & 0.437315 \\
        30.437315 & 0.500153 & -0.087161 & 30.439064 & 0.001750 \\
        30.439064 & 0.500000 & -0.087156 & 30.439065 & 0.000000 \\
        \bottomrule
    \end{tabular}
\end{table}

Based on the table \ref{tab:newton_iterations},
the Newton method converges to a strike of approximately \( K \approx 30.439065 \)
after three iterations, starting from an ATM strike of \( K = 30 \).

\section{exercise 2 Daniel}
% Definitions and formulas for Theta of a European put

Let \( x = \frac{S}{K} \). Then the Theta becomes:
\[
    \Theta(P)(x) = -\frac{\sigma K x}{2\sqrt{2\pi T}} e^{-\frac{d_1^2}{2}} + rK e^{-rT} N(-d_2)
\]
where
\[
    d_1 = \frac{\ln(x) + \left(r + \frac{\sigma^2}{2} \right)T}{\sigma\sqrt{T}}, \quad d_2 = d_1 - \sigma\sqrt{T}
\]

Then the derivative of \( \Theta(P) \) with respect to \( x \) is:
\[
    \frac{d\Theta}{dx} = -\frac{\sigma K}{2\sqrt{2\pi T}} e^{-\frac{d_1^2}{2}}
    - \frac{\sigma K x}{2\sqrt{2\pi T}} e^{-\frac{d_1^2}{2}} \cdot (-d_1) \cdot \frac{1}{x \sigma \sqrt{T}}
    + rK e^{-rT} \cdot \frac{1}{\sqrt{2\pi}} e^{-\frac{d_2^2}{2}} \cdot \left(-\frac{1}{x \sigma \sqrt{T}} \right)
\]

\begin{table}
    \centering
    \caption{Newton's Method for Theta Function}
    \label{tab:newton_theta}
    \begin{tabular}{rrrrrr}
        \toprule
        iteration & x        & f(x)      & f'(x)     & x\_new   & error    \\
        \midrule
        0         & 0.700000 & 0.009182  & -0.239005 & 0.738419 & 0.038419 \\
        1         & 0.738419 & -0.001072 & -0.292386 & 0.734752 & 0.003667 \\
        2         & 0.734752 & -0.000008 & -0.287984 & 0.734724 & 0.000028 \\
        3         & 0.734724 & -0.000000 & -0.287950 & 0.734724 & 0.000000 \\
        \bottomrule
    \end{tabular}
\end{table}

The Newton's method converges to \( S/K \approx 0.734724 \) after three iterations,
starting from \( x = 0.7 \) based on the table \ref{tab:newton_theta}.

\section{exercise 3 Daniel}
\[
    \frac{\partial C}{\partial \sigma} = \text{Vega} = S e^{-qT} \phi(d_1) \sqrt{T}
\]

\[
    d_1 = \frac{\ln(S/K) + (r - q + \frac{1}{2} \sigma^2)T}{\sigma \sqrt{T}}
\]

\[
    \sigma_{n+1} = \sigma_n - f(\sigma_n) \cdot \frac{\sigma_n - \sigma_{n-1}}{f(\sigma_n) - f(\sigma_{n-1})}
\]


\begin{table}
    \centering
    \caption{Newton-Raphson Method Results}
    \label{tab:newton_results}
    \begin{tabular}{rrrrrr}
        \toprule
        iter & sigma    & price    & error    & vega (slope) & next\_sigma \\
        \midrule
        0    & 0.500000 & 6.194813 & 1.694813 & 7.968646     & 0.287315    \\
        1    & 0.287315 & 4.552267 & 0.052267 & 7.262646     & 0.280118    \\
        2    & 0.280118 & 4.500202 & 0.000202 & 7.205791     & 0.280090    \\
        3    & 0.280090 & 4.500000 & 0.000000 & 7.205562     & 0.280090    \\
        \bottomrule
    \end{tabular}
\end{table}

In the Newton method, the implied volatility converges to approximately \( \sigma \approx 0.280090 \)
after three iterations, starting from \( \sigma = 0.5 \) based on table \ref{tab:newton_results}.

\begin{table}
    \centering
    \caption{Secant Method Results}
    \label{tab:secant_results}
    \begin{tabular}{rrrrr}
        \toprule
        iter & sigma    & error    & slope    & next\_sigma \\
        \midrule
        0    & 0.500000 & 1.694813 & NaN      & 0.501000    \\
        1    & 0.501000 & 1.702782 & 7.969178 & 0.287329    \\
        2    & 0.287329 & 0.052370 & 7.724082 & 0.280549    \\
        3    & 0.280549 & 0.003307 & 7.236335 & 0.280092    \\
        4    & 0.280092 & 0.000013 & 7.207446 & 0.280090    \\
        \bottomrule
    \end{tabular}
\end{table}
The secant method converges to approximately \( \sigma \approx 0.280090 \)
after four iterations, starting from \( \sigma = 0.5 \) based on table \ref{tab:secant_results},
which is consistent with the Newton-Raphson method results.

\section{exercise 4 Daniel}
% Black-Scholes Call Price with Continuous Dividend Yield
\[
    C(S, K, T, r, q, \sigma) = S e^{-qT} N(d_1) - K e^{-rT} N(d_2)
\]

% Definitions of d_1 and d_2
\[
    d_1 = \frac{\ln(S/K) + (r - q + \frac{1}{2}\sigma^2)T}{\sigma \sqrt{T}}, \quad
    d_2 = d_1 - \sigma \sqrt{T}
\]

% Vega of the option
\[
    \text{Vega}(\sigma) = S e^{-qT} \sqrt{T} \cdot \frac{1}{\sqrt{2\pi}} e^{-d_1^2 / 2}
\]

% Function whose root gives the implied volatility
\[
    f(\sigma) = C(S, K, T, r, q, \sigma) - C_{\text{market}}
\]


\begin{table}
    \caption{Convergence of methods for implied volatility}
    \label{tab:implied_volatility_convergence}
    \begin{tabular}{rrr}
        \toprule
        Bisection & Secant    & Newton    \\
        \midrule
        0.5000500 & 0.5000000 & 0.5000000 \\
        0.2500750 & 0.4900000 & 0.2556527 \\
        0.3750625 & 0.2557150 & 0.2569032 \\
        0.3125687 & 0.2569076 & 0.2569032 \\
        0.2813219 & 0.2569032 & NaN       \\
        0.2656984 & 0.2569032 & NaN       \\
        0.2578867 & NaN       & NaN       \\
        0.2539809 & NaN       & NaN       \\
        0.2559338 & NaN       & NaN       \\
        0.2569103 & NaN       & NaN       \\
        0.2564220 & NaN       & NaN       \\
        0.2566661 & NaN       & NaN       \\
        0.2567882 & NaN       & NaN       \\
        0.2568492 & NaN       & NaN       \\
        0.2568797 & NaN       & NaN       \\
        0.2568950 & NaN       & NaN       \\
        0.2569026 & NaN       & NaN       \\
        0.2569064 & NaN       & NaN       \\
        0.2569045 & NaN       & NaN       \\
        \bottomrule
    \end{tabular}
\end{table}

We can see from the table \ref{tab:implied_volatility_convergence} that all three methods converge to the same implied volatility of approximately \( \sigma \approx 0.256903 \).

% Approximate implied volatility formula
\[
    \sigma_{\text{imp, approx}} \approx
    \frac{ \sqrt{2\pi} \left( C - \frac{(r - q)T}{2} S \right) }
    { S \sqrt{T} \left( 1 - \frac{(r + q)T}{2} \right) }
\]

% Relative error formula
\[
    \text{Relative Error} =
    \frac{ \left| \sigma_{\text{imp, approx}} - \sigma_{\text{imp}} \right| }
    { \sigma_{\text{imp}} }
\]
\[
    \sigma_{\text{imp, approx}} = 0.256710, \quad
    \text{Relative Error} = 0.00075
\]
Impressive!

\section{exercise 5}
\section{exercise 6}
\section{exercise 7}
\section{exercise 8}
\section{exercise 9}
\section{exercise 10}
\section{exercise 11}
\section{exercise 12}
\section{exercise 13}
\section{exercise 14}
\section{exercise 15}
\section{exercise 16}
\section{exercise 17}
\section{exercise 18 Daniel}
C.

The sensitivity of the European call and put option values with respect to the interest rate \( r \) (known as rho) are:

\[
    \rho_{\text{call}} = \frac{\partial C}{\partial r} = T K e^{-rT} N(d_2)
\]

\[
    \rho_{\text{put}} = \frac{\partial P}{\partial r} = -T K e^{-rT} N(-d_2)
\]

where

\[
    d_2 = \frac{\ln(S/K) + (r - q - \tfrac{1}{2} \sigma^2) T}{\sigma \sqrt{T}}
\]

So the value of the call increases with the interest rate, while for the put, it decreases.

\section{exercise 19 Daniel}
B.

Delta of a share is 1, if we short, it is -1, and if we have 10, it is -10.

The Delta of a put ATM is around -0.5, and if we short it, it is 0.5, and if we have 10, it is 5.

The sum is -5.
\end{document}
