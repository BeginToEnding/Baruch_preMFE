\documentclass{article}
\usepackage{amsmath}
\usepackage{amssymb}
\usepackage{array}
\usepackage{graphicx} % Required for \scalebox
\usepackage{hyperref}
\usepackage{listings}
\usepackage{xcolor}
\usepackage{graphicx}
\usepackage{booktabs}
\usepackage{verbatim}
\usepackage[extreme]{savetrees} % tighter margins
\usepackage{lmodern} % Use scalable Latin Modern fonts
\title{Baruch ACDE HW 4}
\author{group 1}

\begin{document}
\maketitle
\tableofcontents

\section{exercise 1 Daniel}
The derivative of the $\Delta$ is needed for the Newton method:

\[
    N(x) = \int_{-\infty}^x \frac{1}{\sqrt{2\pi}} e^{-t^2/2} \, dt
\]

\[
    \frac{d}{dK} \Delta_{\text{call}}(K)
    = \frac{d}{dK} \left( e^{-qT} N(d_1) \right)
    = e^{-qT} \cdot \frac{dN(d_1)}{dK}
    = e^{-qT} \cdot \frac{1}{\sqrt{2\pi}} e^{-d_1^2/2} \cdot \frac{d d_1}{dK}
\]

\[
    d_1 = \frac{\ln(S/K) + (r - q + \frac{1}{2} \sigma^2)T}{\sigma \sqrt{T}}, \quad
    \Rightarrow \frac{d d_1}{dK} = -\frac{1}{K \sigma \sqrt{T}}
\]

\[
    \Rightarrow \frac{d}{dK} \Delta_{\text{call}}(K)
    = -\frac{e^{-qT}}{K \sigma \sqrt{2\pi T}} e^{-d_1^2/2}
\]

\begin{table}
    \centering
    \caption{Newton Method Iterations for Finding Strike starting from ATM}
    \label{tab:newton_iterations}
    \begin{tabular}{rrrrr}
        \toprule
        K         & Delta(K) & dDelta/dK & Next K    & Error    \\
        \midrule
        30.000000 & 0.538480 & -0.087991 & 30.437315 & 0.437315 \\
        30.437315 & 0.500153 & -0.087161 & 30.439064 & 0.001750 \\
        30.439064 & 0.500000 & -0.087156 & 30.439065 & 0.000000 \\
        \bottomrule
    \end{tabular}
\end{table}

Based on the table \ref{tab:newton_iterations},
the Newton method converges to a strike of approximately \( K \approx 30.439065 \)
after three iterations, starting from an ATM strike of \( K = 30 \).

\section{exercise 2 Daniel}
% Definitions and formulas for Theta of a European put

Let \( x = \frac{S}{K} \). Then the Theta becomes:
\[
    \Theta(P)(x) = -\frac{\sigma K x}{2\sqrt{2\pi T}} e^{-\frac{d_1^2}{2}} + rK e^{-rT} N(-d_2)
\]
where
\[
    d_1 = \frac{\ln(x) + \left(r + \frac{\sigma^2}{2} \right)T}{\sigma\sqrt{T}}, \quad d_2 = d_1 - \sigma\sqrt{T}
\]

Then the derivative of \( \Theta(P) \) with respect to \( x \) is:
\[
    \frac{d\Theta}{dx} = -\frac{\sigma K}{2\sqrt{2\pi T}} e^{-\frac{d_1^2}{2}}
    - \frac{\sigma K x}{2\sqrt{2\pi T}} e^{-\frac{d_1^2}{2}} \cdot (-d_1) \cdot \frac{1}{x \sigma \sqrt{T}}
    + rK e^{-rT} \cdot \frac{1}{\sqrt{2\pi}} e^{-\frac{d_2^2}{2}} \cdot \left(-\frac{1}{x \sigma \sqrt{T}} \right)
\]

\begin{table}
    \centering
    \caption{Newton's Method for Theta Function}
    \label{tab:newton_theta}
    \begin{tabular}{rrrrrr}
        \toprule
        iteration & x        & f(x)      & f'(x)     & x\_new   & error    \\
        \midrule
        0         & 0.700000 & 0.009182  & -0.239005 & 0.738419 & 0.038419 \\
        1         & 0.738419 & -0.001072 & -0.292386 & 0.734752 & 0.003667 \\
        2         & 0.734752 & -0.000008 & -0.287984 & 0.734724 & 0.000028 \\
        3         & 0.734724 & -0.000000 & -0.287950 & 0.734724 & 0.000000 \\
        \bottomrule
    \end{tabular}
\end{table}

The Newton's method converges to \( S/K \approx 0.734724 \) after three iterations,
starting from \( x = 0.7 \) based on the table \ref{tab:newton_theta}.

\section{exercise 3 Daniel}
\[
    \frac{\partial C}{\partial \sigma} = \text{Vega} = S e^{-qT} \phi(d_1) \sqrt{T}
\]

\[
    d_1 = \frac{\ln(S/K) + (r - q + \frac{1}{2} \sigma^2)T}{\sigma \sqrt{T}}
\]

\[
    \sigma_{n+1} = \sigma_n - f(\sigma_n) \cdot \frac{\sigma_n - \sigma_{n-1}}{f(\sigma_n) - f(\sigma_{n-1})}
\]


\begin{table}
    \centering
    \caption{Newton-Raphson Method Results}
    \label{tab:newton_results}
    \begin{tabular}{rrrrrr}
        \toprule
        iter & sigma    & price    & error    & vega (slope) & next\_sigma \\
        \midrule
        0    & 0.500000 & 6.194813 & 1.694813 & 7.968646     & 0.287315    \\
        1    & 0.287315 & 4.552267 & 0.052267 & 7.262646     & 0.280118    \\
        2    & 0.280118 & 4.500202 & 0.000202 & 7.205791     & 0.280090    \\
        3    & 0.280090 & 4.500000 & 0.000000 & 7.205562     & 0.280090    \\
        \bottomrule
    \end{tabular}
\end{table}

In the Newton method, the implied volatility converges to approximately \( \sigma \approx 0.280090 \)
after three iterations, starting from \( \sigma = 0.5 \) based on table \ref{tab:newton_results}.

\begin{table}
    \centering
    \caption{Secant Method Results}
    \label{tab:secant_results}
    \begin{tabular}{rrrrr}
        \toprule
        iter & sigma    & error    & slope    & next\_sigma \\
        \midrule
        0    & 0.500000 & 1.694813 & NaN      & 0.501000    \\
        1    & 0.501000 & 1.702782 & 7.969178 & 0.287329    \\
        2    & 0.287329 & 0.052370 & 7.724082 & 0.280549    \\
        3    & 0.280549 & 0.003307 & 7.236335 & 0.280092    \\
        4    & 0.280092 & 0.000013 & 7.207446 & 0.280090    \\
        \bottomrule
    \end{tabular}
\end{table}
The secant method converges to approximately \( \sigma \approx 0.280090 \)
after four iterations, starting from \( \sigma = 0.5 \) based on table \ref{tab:secant_results},
which is consistent with the Newton-Raphson method results.

\section{exercise 4 Daniel}
% Black-Scholes Call Price with Continuous Dividend Yield
\[
    C(S, K, T, r, q, \sigma) = S e^{-qT} N(d_1) - K e^{-rT} N(d_2)
\]

% Definitions of d_1 and d_2
\[
    d_1 = \frac{\ln(S/K) + (r - q + \frac{1}{2}\sigma^2)T}{\sigma \sqrt{T}}, \quad
    d_2 = d_1 - \sigma \sqrt{T}
\]

% Vega of the option
\[
    \text{Vega}(\sigma) = S e^{-qT} \sqrt{T} \cdot \frac{1}{\sqrt{2\pi}} e^{-d_1^2 / 2}
\]

% Function whose root gives the implied volatility
\[
    f(\sigma) = C(S, K, T, r, q, \sigma) - C_{\text{market}}
\]


\begin{table}
    \caption{Convergence of methods for implied volatility}
    \label{tab:implied_volatility_convergence}
    \begin{tabular}{rrr}
        \toprule
        Bisection & Secant    & Newton    \\
        \midrule
        0.5000500 & 0.5000000 & 0.5000000 \\
        0.2500750 & 0.4900000 & 0.2556527 \\
        0.3750625 & 0.2557150 & 0.2569032 \\
        0.3125687 & 0.2569076 & 0.2569032 \\
        0.2813219 & 0.2569032 & NaN       \\
        0.2656984 & 0.2569032 & NaN       \\
        0.2578867 & NaN       & NaN       \\
        0.2539809 & NaN       & NaN       \\
        0.2559338 & NaN       & NaN       \\
        0.2569103 & NaN       & NaN       \\
        0.2564220 & NaN       & NaN       \\
        0.2566661 & NaN       & NaN       \\
        0.2567882 & NaN       & NaN       \\
        0.2568492 & NaN       & NaN       \\
        0.2568797 & NaN       & NaN       \\
        0.2568950 & NaN       & NaN       \\
        0.2569026 & NaN       & NaN       \\
        0.2569064 & NaN       & NaN       \\
        0.2569045 & NaN       & NaN       \\
        \bottomrule
    \end{tabular}
\end{table}

We can see from the table \ref{tab:implied_volatility_convergence} that all three methods converge to the same implied volatility of approximately \( \sigma \approx 0.256903 \).

% Approximate implied volatility formula
\[
    \sigma_{\text{imp, approx}} \approx
    \frac{ \sqrt{2\pi} \left( C - \frac{(r - q)T}{2} S \right) }
    { S \sqrt{T} \left( 1 - \frac{(r + q)T}{2} \right) }
\]

% Relative error formula
\[
    \text{Relative Error} =
    \frac{ \left| \sigma_{\text{imp, approx}} - \sigma_{\text{imp}} \right| }
    { \sigma_{\text{imp}} }
\]
\[
    \sigma_{\text{imp, approx}} = 0.256710, \quad
    \text{Relative Error} = 0.00075
\]
Impressive!

\section{exercise 5 Hao}
% Black-Scholes Call and Put Price with Continuous Dividend Yield
\[
    C(S, K, T, r, q, \sigma) = S e^{-qT} N(d_1) - K e^{-rT} N(d_2)
\]

\[
    P(S, K, T, r, q, \sigma) = K e^{-rT} N(-d_2) - S e^{-qT} N(-d_1)
\]

% Definitions of d_1 and d_2
\[
    d_1 = \frac{\ln(S/K) + (r - q + \frac{1}{2}\sigma^2)T}{\sigma \sqrt{T}}, \quad
    d_2 = d_1 - \sigma \sqrt{T}
\]

\[
    \text{Vega}(\sigma) = S e^{-qT} \sqrt{T} \cdot \frac{1}{\sqrt{2\pi}} e^{-d_1^2 / 2}
\]

% Function whose root gives the implied volatility
\[
    f_C(\sigma) = C(S, K, T, r, q, \sigma) - C_{\text{market}}
\]

\[
    f_P(\sigma) = P(S, K, T, r, q, \sigma) - P_{\text{market}}
\]

Given \(S=1370, T = 199/252, r=0.04567, q=0.019054 \),
we use Newton's Method and Explicit Approximation to estimate the implied volatility,
the result is shown in table 6


\begin{table}[htbp]
    \centering
    \caption{Call Options: Implied Volatility Comparison}
    \label{tab:call_iv}
    \begin{tabular}{lccc}
        \toprule
        Strike & {Newton IV} & {Approx IV} & {Error (\%)} \\
        \midrule
        C1175  & 0.257273    & 1.411213    & 448.53       \\
        C1200  & 0.249599    & 1.236640    & 395.45       \\
        C1225  & 0.241899    & 1.056759    & 336.86       \\
        C1250  & 0.234400    & 0.869813    & 271.08       \\
        C1275  & 0.226261    & 0.677791    & 199.56       \\
        C1300  & 0.218605    & 0.475593    & 117.56       \\
        C1325  & 0.211442    & 0.263115    & 24.44        \\
        C1350  & 0.204054    & 0.204040    & -0.01        \\
        C1375  & 0.196911    & 0.194043    & -1.46        \\
        C1400  & 0.189418    & 0.189243    & -0.09        \\
        C1425  & 0.182563    & 0.182151    & -0.23        \\
        C1450  & 0.175229    & 0.174466    & -0.44        \\
        C1500  & 0.163359    & 0.161552    & -1.11        \\
        C1550  & 0.150488    & 0.147163    & -2.21        \\
        C1575  & 0.144821    & 0.140620    & -2.90        \\
        C1600  & 0.141285    & 0.136237    & -3.57        \\
        \bottomrule
    \end{tabular}
\end{table}

\begin{table}[htbp]
    \centering
    \caption{Put Options: Implied Volatility Comparison}
    \label{tab:put_iv}
    \begin{tabular}{rrrrr}
        \toprule
        Strike & {Newton IV} & {Approx IV} & {Error (\%)} \\
        \midrule
        P1175  & 0.257192    & 0.254945    & -0.87        \\
        P1200  & 0.249237    & 0.247547    & -0.68        \\
        P1225  & 0.241654    & 0.240447    & -0.50        \\
        P1250  & 0.234024    & 0.233223    & -0.34        \\
        P1275  & 0.226557    & 0.226080    & -0.21        \\
        P1300  & 0.219088    & 0.218853    & -0.11        \\
        P1325  & 0.211998    & 0.211919    & -0.04        \\
        P1350  & 0.204371    & 0.040477    & -80.19       \\
        P1375  & 0.196682    & 0.196637    & -0.02        \\
        P1400  & 0.189482    & 0.442855    & 133.72       \\
        P1425  & 0.182499    & 0.706597    & 287.18       \\
        P1450  & 0.176812    & 0.980996    & 454.82       \\
        P1500  & 0.162441    & 1.609792    & 891.00       \\
        P1550  & 0.150879    & 2.315074    & 1434.39      \\
        P1575  & 0.144861    & 2.716735    & 1775.41      \\
        P1600  & 0.140352    & 3.119543    & 2122.66      \\
        \bottomrule
    \end{tabular}
\end{table}

\begin{enumerate}
    \item The explicit approximation provides satisfactory accuracy for put options, making it suitable for real-time applications where speed is critical.

    \item For call options, particularly OTM contracts, the approximation exhibits significant bias, suggesting Newton's method remains preferable when accuracy is paramount.

    \item The inverse relationship between approximation error and moneyness indicates the formula's limitation in capturing volatility smile effects at extreme strikes.
\end{enumerate}

\section{exercise 6 Hao}

Find all local extremum points of the function:
\[ f(x,y) = 3x^{2}y + x^{2} - 6x - 3y - 2 \]

\subsection*{Step 1: Compute First Partial Derivatives}
\begin{align*}
    \frac{\partial f}{\partial x} & = 6xy + 2x - 6 \\
    \frac{\partial f}{\partial y} & = 3x^{2} - 3
\end{align*}

\subsection*{Step 2: Find Critical Points}
Set partial derivatives equal to zero:
\[
    \begin{cases}
        6xy + 2x - 6 = 0 \quad (1) \\
        3x^{2} - 3 = 0 \quad (2)
    \end{cases}
\]

From equation (2):
\[ 3x^{2} - 3 = 0 \implies x^{2} = 1 \implies x = \pm 1 \]

\subsubsection*{Case 1: x = 1}
Substitute into (1):
\[ 6(1)y + 2(1) - 6 = 0 \implies 6y - 4 = 0 \implies y = \frac{2}{3} \]

\subsubsection*{Case 2: x = -1}
Substitute into (1):
\[ 6(-1)y + 2(-1) - 6 = 0 \implies -6y - 8 = 0 \implies y = -\frac{4}{3} \]

Critical points: $(1, \frac{2}{3})$ and $(-1, -\frac{4}{3})$

\subsection*{Step 3: Second Derivative Test}
Compute second partial derivatives:
\begin{align*}
    \frac{\partial^{2} f}{\partial x^{2}}        & = 6y + 2 \\
    \frac{\partial^{2} f}{\partial y^{2}}        & = 0      \\
    \frac{\partial^{2} f}{\partial x \partial y} & = 6x
\end{align*}

Hessian matrix determinant:
\[ D = f_{xx}f_{yy} - (f_{xy})^{2} = (6y + 2)(0) - (6x)^{2} = -36x^{2} \]

\subsubsection*{At $(1, \frac{2}{3})$}
\[ D = -36(1)^{2} = -36 < 0 \implies \text{Saddle point} \]

\subsubsection*{At $(-1, -\frac{4}{3})$}
\[ D = -36(-1)^{2} = -36 < 0 \implies \text{Saddle point} \]

The function has no local extrema, only saddle points at $(1, \frac{2}{3})$ and $(-1, -\frac{4}{3})$.


\section{exercise 7 Hao}
Find the maximum and minimum of:
\[ f(\mathbf{x}) = x_{1}^{2} + 3x_{2}^{2} + x_{3}^{2} - 2x_{1}x_{2} - x_{2}x_{3} + x_{1}x_{3} \]
subject to $x_{1}^{2} + x_{2}^{2} + x_{3}^{2} = 6$.

\subsection*{Step 1: Lagrange Multiplier Setup}
Define the Lagrangian:
\[ \mathcal{L} = f(\mathbf{x}) - \lambda(x_{1}^{2} + x_{2}^{2} + x_{3}^{2} - 6) \]

\subsection*{Step 2: Take Partial Derivatives}
\begin{align*}
    \frac{\partial \mathcal{L}}{\partial x_{1}} & = 2x_{1} - 2x_{2} + x_{3} - 2\lambda x_{1} = 0 \\
    \frac{\partial \mathcal{L}}{\partial x_{2}} & = 6x_{2} - 2x_{1} - x_{3} - 2\lambda x_{2} = 0 \\
    \frac{\partial \mathcal{L}}{\partial x_{3}} & = 2x_{3} - x_{2} + x_{1} - 2\lambda x_{3} = 0
\end{align*}

\subsection*{Step 3: Rewrite as Eigenvalue Problem}
This is equivalent to solving:
\[ A\mathbf{x} = \lambda\mathbf{x} \]
where:
\[ A = \begin{pmatrix}
        1   & -1   & 0.5  \\
        -1  & 3    & -0.5 \\
        0.5 & -0.5 & 1
    \end{pmatrix} \]

\subsection*{Step 4: Find Eigenvalues}  ·
Characteristic equation:
\[ \det(A - \lambda I) = -\lambda^{3} + 5\lambda^{2} - 5.5\lambda + 1.5 = 0 \]

Solutions: $\lambda = 1, 2+\frac{\sqrt{10}}{2}, 2-\frac{\sqrt{10}}{2}$

\subsection*{Step 5: Find Corresponding Points}
\subsubsection*{Case $\lambda = 1$}
Solve $(A - \lambda I)\mathbf{x} = 0$:
\[ \begin{cases}
        -2x_2 + x_3 = 0        \\
        -2x_1 + 4x_2 - x_3 = 0 \\
        x_1 - x_2 = 0
    \end{cases} \]
Solution: $x_1 = x_2 = t$, $x_3 = 2t$

Normalize to satisfy constraint:
\[ 6t^2 = 6 \implies t = \pm 1 \]
Points: $\pm(1, 1, 2)$

\subsubsection*{Case $\lambda = 2+\frac{\sqrt{10}}{2}$}
Solve $(A - \lambda I)\mathbf{x} = 0$:
\[ \begin{cases}
        -(2+\sqrt{10})x_1 - 2x_2 + x_3 = 0 \\
        -2x_1 +(2-\sqrt{10})x_2 - x_3 = 0  \\
        x_1 - x_2 - (2+\sqrt{10})x_3 = 0
    \end{cases} \]
Solution: $x_1 = \frac{\sqrt{10}}{2}t$, $x_2 = (-\frac{\sqrt{10}}{2}-2)t$, $x_3 = t$

Normalize to satisfy constraint:
Points: $\pm(0.960, -2.174, 0.607)$

\subsubsection*{Case $\lambda = 2-\frac{\sqrt{10}}{2}$}
Solve $(A - \lambda I)\mathbf{x} = 0$:
\[ \begin{cases}
        (\sqrt{10}-2)x_1 - 2x_2 + x_3 = 0 \\
        -2x_1 +(2+\sqrt{10})x_2 - x_3 = 0 \\
        x_1 - x_2 +(\sqrt{10}-2)x_3 = 0
    \end{cases} \]
Solution: $x_1 = -\frac{\sqrt{10}}{2}t$, $x_2 = (\frac{\sqrt{10}}{2}-2)t$, $x_3 = t$


Normalize to satisfy constraint:
Points: $\pm(-2.019, -0.535, 1.277)$

\subsection*{Step 6: Evaluate Function}
\begin{align*}
    f(1,1,2) = 9                        \\
    f(0.960, -2.174, 0.607) = 21.545295 \\
    f(-2.019, -0.535, 1.277) = 2.510367 \\
\end{align*}

\subsection*{Conclusion}
The function has:
\begin{itemize}
    \item Minimum value: 2.510367
    \item Maximum value: 21.545295
\end{itemize}
on the given constraint surface.


\section{exercise 8 Hongjun}
C.\\
At-the-money options have the highest time value, so their premiums exceed those of out-of-the-money options.

\section{exercise 9 Hongjun}
B.\\
Buying both a 95-call and a 95-put (a straddle) caps your worst-case loss at the total premiums paid: \$8 + \$5 = \$13 when St = 95.

\section{exercise 10 Hongjun}
C.\\
Selling naked options carries unlimited risk if the market moves sharply against you, making it the most speculative.

\section{exercise 11 Hongjun}
A.\\
A put seller loses when the underlying falls; the greater the price drop, the larger the loss.

\section{exercise 12 Hongjun}
A.\\
Total premium received = \$62 + \$15 = \$77, so breakevens are 3025 - 77 = 2948 and 3050 + 77 = 3127.

\section{exercise 13 Hongjun}
C.\\
Long put + short call at the same strike replicates a synthetic short stock position.

\section{exercise 14 Hongjun}
D.\\
A long put butterfly profits if the index stays near its current level and limits losses on a substantial down move.

\section{exercise 15 Hongjun}
B.\\
A 1×2 put spread is typically entered for a net credit or zero cost, since you sell more premium than you pay.

\section{exercise 16 Hongjun}
B.\\
The 1×2 call spread outperforms the single call and the plain spread when the underlying finishes between the two strikes.

\section{exercise 17 Hongjun}
C.\\
At-the-money options maximize time value, because it may rise or fall. In other words, it’s vega is the largest, sensitive to volatility.

\section{exercise 18 Daniel}
C.

The sensitivity of the European call and put option values with respect to the interest rate \( r \) (known as rho) are:

\[
    \rho_{\text{call}} = \frac{\partial C}{\partial r} = T K e^{-rT} N(d_2)
\]

\[
    \rho_{\text{put}} = \frac{\partial P}{\partial r} = -T K e^{-rT} N(-d_2)
\]

where

\[
    d_2 = \frac{\ln(S/K) + (r - q - \tfrac{1}{2} \sigma^2) T}{\sigma \sqrt{T}}
\]

So the value of the call increases with the interest rate, while for the put, it decreases.

\section{exercise 19 Daniel}
B.

Delta of a share is 1, if we short, it is -1, and if we have 10, it is -10.

The Delta of a put ATM is around -0.5, and if we short it, it is 0.5, and if we have 10, it is 5.

The sum is -5.
\end{document}
